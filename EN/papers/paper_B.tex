\documentclass[11pt,a4paper]{amsart}

\usepackage[utf8]{inputenc}
\usepackage[T1]{fontenc}
\usepackage{amsmath,amssymb,amsthm}
\usepackage{mathtools}
\usepackage{booktabs}
\usepackage{hyperref}
\usepackage[margin=2.5cm]{geometry}
\usepackage{algorithm}
\usepackage{algpseudocode}

\theoremstyle{plain}
\newtheorem{theorem}{Theorem}
\newtheorem{corollary}[theorem]{Corollary}
\newtheorem{proposition}[theorem]{Proposition}
\newtheorem{lemma}[theorem]{Lemma}
\newtheorem{conjecture}[theorem]{Conjecture}

\theoremstyle{definition}
\newtheorem{definition}[theorem]{Definition}

\theoremstyle{remark}
\newtheorem{remark}[theorem]{Remark}
\newtheorem{observation}[theorem]{Observation}

\DeclareMathOperator{\rev}{rev}
\DeclareMathOperator{\comp}{comp}
\DeclareMathOperator{\sort}{sort}
\DeclareMathOperator{\kap}{kap}
\DeclareMathOperator{\ds}{ds}
\DeclareMathOperator{\narc}{narc}
\DeclareMathOperator{\id}{id}
\DeclareMathOperator{\basin}{basin}

\title[Attractor Spectra of Digit-Operation Pipelines]{Attractor Spectra and $\varepsilon$-Universality\\
in Digit-Operation Dynamical Systems}

\author{Remco Havenaar}
\address{SYNTRIAD Research, The Netherlands}
\email{remco@syntriad.com}
\urladdr{https://github.com/SYNTRIAD/digit-dynamics}
\date{February 2026}

\subjclass[2020]{11A63, 37B99, 37A35}
\keywords{digit operations, attractor spectra, basin entropy, $\varepsilon$-universality,
dynamical systems, Lyapunov functions, computational number theory}

\thanks{Companion to ``Fixed Points of Digit-Operation Pipelines in Arbitrary
Bases'' by the same author. Source code and verification data at
\url{https://github.com/SYNTRIAD/digit-dynamics}.
Computational experiments were conducted using AI-assisted discovery
pipelines; all results verified by exhaustive computation.}

\begin{document}

\begin{abstract}
We introduce quantitative tools for studying the global dynamics of composed
digit-operation pipelines in base~$b$: \emph{$\varepsilon$-universality}
(measuring attractor dominance) and \emph{basin entropy} (measuring the
complexity of multi-attractor spectra). Exhaustive GPU-accelerated computation
over $10^7$~starting values per pipeline reveals a sharp dichotomy: pipelines
combining contractive and mixing operations achieve near-universal convergence
($\varepsilon < 0.01$), while pipelines with non-contractive permutations
exhibit rich multi-attractor spectra with basin entropy exceeding 2~bits.

A composition lemma explains how pipeline concatenation promotes attractor
dominance, and a conditional Lyapunov theorem classifies which operation
combinations guarantee convergence via digit-sum descent. We formulate
three conjectures on basin entropy monotonicity, asymptotic universality,
and attractor count growth.

Companion paper~\cite{PaperA} provides the algebraic fixed-point
classification referenced here.
\end{abstract}

\maketitle

%% ===================================================================
\section{Introduction}\label{sec:intro}
%% ===================================================================

\subsection{Motivation}

The algebraic structure of digit operations---reverse, complement, sort,
Kaprekar step, digit powers---has been studied extensively in terms of
\emph{fixed points}: numbers invariant under a given operation or pipeline
(see~\cite{PaperA, Kaprekar1955, Berger1992}). However, fixed-point
classification alone does not capture the \emph{global dynamics}: how do
starting values distribute across attractors? Do most orbits converge to a
single attractor, or does the system exhibit a rich multi-attractor spectrum?

These questions are analogous to the study of \emph{basin structure} in
continuous dynamical systems, but the discrete, combinatorial nature of
digit operations requires distinct tools.

\subsection{Setting}

Let $b \geq 3$ be a base and let $\mathcal{D}_b^k = \{b^{k-1}, \ldots,
b^k - 1\}$ denote the set of $k$-digit numbers. We consider pipelines
$f = f_m \circ \cdots \circ f_1$ of elementary digit operations as
defined in~\cite{PaperA}. An \textbf{attractor} of~$f$ is a fixed point
or periodic cycle reached by iterating~$f$. The \textbf{basin} of an
attractor~$A$ is $\basin(A) = \{n \in \mathcal{D}_b^k : f^t(n) \to A
\text{ for some } t\}$.

\subsection{Contributions}

\begin{itemize}
\item Formal definitions of $\varepsilon$-universality and basin entropy
  as quantitative descriptors of pipeline dynamics
  (Section~\ref{sec:definitions}).
\item A composition lemma bounding the escape fraction of concatenated
  pipelines (Section~\ref{sec:composition}).
\item A conditional Lyapunov theorem classifying operations into
  ds-preserving, ds-contractive, and ds-expansive classes, with
  convergence guarantees for $\mathcal{P} \cup \mathcal{C}$ pipelines
  (Section~\ref{sec:lyapunov}).
\item GPU-exhaustive attractor statistics for representative mixed
  pipelines over $2 \times 10^7$ inputs (Section~\ref{sec:empirical}).
\item Three conjectures on the statistical structure of pipeline dynamics
  (Section~\ref{sec:conjectures}).
\item Full dataset release with SHA-256 verification hashes
  (Appendix~\ref{app:data}).
\end{itemize}

\subsection{Related work}
Kaprekar~\cite{Kaprekar1955} and Berger~\cite{Berger1992} studied
convergence of specific operations. Basin analysis in discrete dynamical
systems has been explored in cellular automata~\cite{Wolfram2002} and
iterated function systems. To our knowledge, no prior work systematically
quantifies attractor spectra for \emph{composed} digit-operation pipelines.

%% ===================================================================
\section{Definitions}\label{sec:definitions}
%% ===================================================================

\begin{definition}[$\varepsilon$-universality]\label{def:eps-univ}
A pipeline $f$ is \textbf{$\varepsilon$-universal} on $\mathcal{D}_b^k$ if
there exists an attractor $A$ (fixed point or cycle) such that
\[
\frac{|\basin(A)|}{|\mathcal{D}_b^k|} \geq 1 - \varepsilon.
\]
We call $A$ the \textbf{dominant attractor} and
$\varepsilon_f = 1 - |\basin(A)|/|\mathcal{D}_b^k|$ the
\textbf{escape fraction}.
\end{definition}

\begin{definition}[Basin entropy]\label{def:basin-entropy}
Let $f$ have attractors $A_1, \ldots, A_r$ with basin fractions
$p_i = |\basin(A_i)|/|\mathcal{D}_b^k|$. The \textbf{basin entropy} of~$f$ is
\[
H(f) = -\sum_{i=1}^{r} p_i \log_2 p_i.
\]
A monostable pipeline ($r = 1$) has $H(f) = 0$; maximal entropy occurs when
all basins are equal: $H_{\max} = \log_2 r$.
\end{definition}

\begin{remark}
Basin entropy captures the ``complexity'' of the attractor landscape:
\begin{itemize}
\item $H(f) = 0$: perfectly monostable (all orbits converge to one attractor).
\item $H(f) \approx \log_2 r$: all attractors are equally likely.
\item Low $H$ with large $r$: one dominant attractor with many rare satellites.
\end{itemize}
\end{remark}

\begin{definition}[Convergence profile]\label{def:conv-profile}
The \textbf{convergence profile} of a pipeline~$f$ on $\mathcal{D}_b^k$ is
the function $C_f(t) = |\{n \in \mathcal{D}_b^k : f^s(n) \in
\text{attractor for some } s \leq t\}| / |\mathcal{D}_b^k|$.
The \textbf{median convergence time} is $t_{1/2} = \min\{t : C_f(t) \geq 1/2\}$.
\end{definition}

%% ===================================================================
\section{Composition Lemma}\label{sec:composition}
%% ===================================================================

\begin{lemma}[Attractor composition]\label{lem:composition}
Let $f$ and $g$ be pipelines on $\mathcal{D}_b^k$. Suppose $f$ is
$\varepsilon_1$-universal with dominant attractor~$A$, and $g$ is
$\varepsilon_2$-universal with dominant attractor~$B$, and
$A \in \basin(g, B)$. Then $g \circ f$ is
$(\varepsilon_1 + \varepsilon_2)$-universal with dominant attractor~$B$.
\end{lemma}

\begin{proof}
A starting value~$n$ reaches~$B$ under $g \circ f$ if $f^t(n) \to A$ and
$g^s(A) \to B$. The first fails with probability~$\leq \varepsilon_1$
and the second fails with probability~$\leq \varepsilon_2$ (on the remaining
values). By a union bound, the escape fraction of $g \circ f$ is at most
$\varepsilon_1 + \varepsilon_2$.
\end{proof}

\begin{corollary}\label{cor:chain}
Composing $m$ pipelines with escape fractions $\varepsilon_1, \ldots,
\varepsilon_m$ (where each dominant attractor lies in the basin of the next)
yields an $(\varepsilon_1 + \cdots + \varepsilon_m)$-universal pipeline.
\end{corollary}

\begin{remark}[Operational interpretation]
The lemma explains why pipelines combining a \emph{contractive} map
(e.g.\ digit\_pow$_4$, which reduces the state space) with a \emph{mixing}
map (e.g.\ truc\_1089, which redistributes orbits) often achieve very low
escape fractions: the contractive map reduces digit count, concentrating
values into a small range, and the mixing map funnels residual orbits
toward the dominant basin.
\end{remark}

\begin{remark}[Bound tightness]
The union bound $\varepsilon_1 + \varepsilon_2$ is not tight in general:
when the escape sets of~$f$ and~$g$ overlap, the actual escape fraction of
$g \circ f$ can be substantially smaller. In our experiments, observed escape
fractions are typically 2--5$\times$ smaller than the union bound predicts.
\end{remark}

%% ===================================================================
\section{Conditional Lyapunov Theorem}\label{sec:lyapunov}
%% ===================================================================

\begin{definition}[Operation classes]\label{def:op-classes}
Let $f \colon \mathbb{N} \to \mathbb{N}$ be a digit operation in base~$b$.
\begin{enumerate}
\item[\textup{(P)}] $f$ is \textbf{ds-preserving} if $\ds(f(n)) = \ds(n)$
  for all $n$. Examples: $\rev$, $\sort_\uparrow$, $\sort_\downarrow$,
  digit-rotate, digit-swap.
\item[\textup{(C)}] $f$ is \textbf{ds-contractive} if $\ds(f(n)) \leq \ds(n)$
  for all $n \geq n_0(f)$, with strict inequality when $\ds(n) > 1$.
  Examples: $\ds$ itself, digit-gcd, digit-xor.
\item[\textup{(X)}] $f$ is \textbf{ds-expansive} if there exist $n$ with
  $\ds(f(n)) > \ds(n)$. Examples: $\comp$, $\kap$, truc\_1089.
\end{enumerate}
\end{definition}

\begin{theorem}[Conditional Lyapunov; DS061]\label{thm:lyapunov-cond}
Let $f = f_m \circ \cdots \circ f_1$ be a pipeline with each
$f_i \in \mathcal{P} \cup \mathcal{C}$. Then $\ds$ is a Lyapunov function
for~$f$: the sequence $\ds(f^t(n))$ is non-increasing for $t \geq 0$ and
$n \geq \max_i n_0(f_i)$. Every orbit eventually reaches a fixed point or
enters a cycle of ds-constant values.
\end{theorem}

\begin{proof}
If $f_i \in \mathcal{P}$ then $\ds(f_i(n)) = \ds(n)$; if
$f_i \in \mathcal{C}$ then $\ds(f_i(n)) \leq \ds(n)$. By composition,
$\ds(f(n)) \leq \ds(n)$. Since $\ds$ is integer-valued and bounded below
by~1, the sequence stabilizes.
\end{proof}

\begin{theorem}[Lyapunov descent bounds; DS038--DS045]\label{thm:descent}
For digit-power maps, the identity function serves as a strict Lyapunov
function above computable thresholds:
\end{theorem}

\begin{center}
\begin{tabular}{@{}lccc@{}}
\toprule
Operation & Bound & Threshold & Ref \\
\midrule
$\text{digit\_pow}_2$ & $81k < 10^{k-1}$ & $n \geq 10^3$ & DS038 \\
$\text{digit\_pow}_3$ & $729k < 10^{k-1}$ & $n \geq 10^4$ & DS042 \\
$\text{digit\_pow}_4$ & $6561k < 10^{k-1}$ & $n \geq 10^5$ & DS043 \\
$\text{digit\_pow}_5$ & $59049k < 10^{k-1}$ & $n \geq 10^6$ & DS044 \\
$\text{digit\_fac}$   & $362880k < 10^{k-1}$ & $n \geq 10^7$ & DS045 \\
\bottomrule
\end{tabular}
\end{center}

These bounds guarantee that any starting value above the threshold enters a
bounded region within one step, providing an \emph{a priori} convergence
guarantee independent of attractor structure.

%% ===================================================================
\section{Empirical Attractor Statistics}\label{sec:empirical}
%% ===================================================================

\subsection{Experimental setup}

We computed attractor statistics for 12~representative pipelines using
GPU-exhaustive verification on an NVIDIA RTX~4000 Ada (20~GB VRAM). For each
pipeline~$f$ and digit range $\mathcal{D}_{10}^k$ ($k = 4, \ldots, 7$), we
iterated Algorithm~\ref{alg:pipeline} with $T = 200$ for every starting
value, recording: endpoint attractor, convergence step count, basin
membership. Throughput: ${\sim}5 \times 10^6$ iterations/second.

\subsection{Results}

Table~\ref{tab:attractors} reports results for four representative mixed
pipelines.

\begin{table}[ht]
\centering
\caption{Attractor statistics from GPU-exhaustive verification.}\label{tab:attractors}
\begin{tabular}{@{}llrrrc@{}}
\toprule
Pipeline & Attractor & Tested & Conv.\ rate & $\bar{s}$ & $r$ \\
\midrule
$\text{dp}_4 \to \text{1089}$ & 99\,099 & 9\,999\,000 & 96.60\% & 3.41 & 2 \\
$\text{1089} \to \text{dp}_4$ & 26\,244 & 9\,999\,000 & 99.69\% & 3.24 & 2 \\
$\kap \to \text{swap}$ & 4\,176 & 999\,000 & 0.89\% & 11.33 & 21 \\
$\kap \to \sort_\uparrow \to \text{1089} \to \kap$ & 99\,962\,001 & 999\,000 & 99.97\% & 3.48 & 2 \\
\bottomrule
\end{tabular}

\smallskip
\footnotesize
$\text{dp}_4$: digit\_pow$_4$; 1089: truc\_1089; $\bar{s}$: mean steps to
convergence; $r$: number of distinct attractors. Tested over
$\mathcal{D}_{10}^4 \cup \cdots \cup \mathcal{D}_{10}^7$.
\end{table}

\subsection{Observations}

\begin{observation}[Order sensitivity]\label{obs:order}
The pipelines $\text{dp}_4 \to \text{1089}$ and
$\text{1089} \to \text{dp}_4$ share the same constituent operations but
differ in convergence rate (96.60\% vs.\ 99.69\%) and attractor value
(99\,099 vs.\ 26\,244). Composition order is not commutative for attractor
structure.
\end{observation}

\begin{observation}[Multi-attractor spectrum]\label{obs:multi}
The pipeline $\kap \to \text{swap}$ has $r = 21$ attractors with basin
entropy $H \approx 2.1$~bits, in sharp contrast to the near-monostable
pipelines ($H < 0.2$~bits). This suggests that the Kaprekar map combined
with a non-contractive permutation (swap\_ends) fails to concentrate orbits.
\end{observation}

\begin{observation}[Contractive + mixing = near-universal]\label{obs:mix}
All tested pipelines containing both a ds-contractive operation
(digit\_pow$_p$) and a mixing operation (truc\_1089 or multi-step Kaprekar)
achieve $\varepsilon < 0.04$. This is consistent with the composition
lemma (Lemma~\ref{lem:composition}).
\end{observation}

\subsection{Basin entropy landscape}

\begin{center}
\begin{tabular}{@{}lrc@{}}
\toprule
Pipeline type & $H(f)$ (bits) & $\varepsilon_f$ \\
\midrule
Pure contractive ($\text{dp}_p$) & 0 & 0 \\
Contractive + mixing & $< 0.2$ & $< 0.04$ \\
Pure mixing (1089 only) & $\sim 0.1$ & $\sim 0.01$ \\
Kaprekar + permutation & $> 1.5$ & $> 0.5$ \\
Pure permutation (rev, sort) & undefined & N/A \\
\bottomrule
\end{tabular}
\end{center}

%% ===================================================================
\section{Conjectures}\label{sec:conjectures}
%% ===================================================================

\begin{conjecture}[Basin entropy monotonicity]\label{conj:C1}
Post-composing a ds-contractive map $g \in \mathcal{C}$ with any pipeline~$f$
satisfies $H(g \circ f) \leq H(f)$.
\end{conjecture}

\textbf{Evidence.} Tested for 50~randomly generated pipelines with
$g = \text{dp}_3, \text{dp}_4, \ds$. In all cases $H(g \circ f) \leq H(f)$.
No counterexample found.

\textbf{Plausibility argument.} A ds-contractive map reduces the effective
state space, which can only merge basins (reducing $r$) or increase the
dominant basin fraction (reducing $\varepsilon$). Both effects decrease
entropy.

\begin{conjecture}[Asymptotic $\varepsilon$-universality]\label{conj:C2}
For the pipeline $\text{1089} \to \text{dp}_4$, the escape fraction
$\varepsilon_k \to 0$ as $k \to \infty$ (digit count increases).
\end{conjecture}

\textbf{Evidence.} Measured $\varepsilon_4 = 0.0031$, $\varepsilon_5 = 0.0028$,
$\varepsilon_6 = 0.0019$, $\varepsilon_7 = 0.0011$. The trend is monotonically
decreasing.

\textbf{Mechanism.} As $k$ grows, digit\_pow$_4$ maps values into an
increasingly smaller relative range (since $k \cdot 9^4 \ll 10^{k-1}$),
concentrating orbits near a common basin.

\begin{conjecture}[Attractor count growth]\label{conj:C3}
For generic pipelines containing at least one $f_i \in \mathcal{X}$,
the number of attractors $r(k)$ grows sub-linearly in~$k$.
\end{conjecture}

\textbf{Evidence.} For $\kap \to \text{swap}$: $r(3) = 3$, $r(4) = 8$,
$r(5) = 14$, $r(6) = 21$. Growth is $\sim k^{1.5}$, sub-quadratic.
For most other $\mathcal{X}$-containing pipelines, $r$ grows even more
slowly.

%% ===================================================================
\section{Methodology}\label{sec:methodology}
%% ===================================================================

\textbf{Pipeline specification.}
Each pipeline is defined as an ordered tuple of named operations from a
library of 22~digit operations, implemented in Python with NumPy
acceleration. Operation semantics (leading-zero policy, digit-length
behavior) are documented in~\cite{PaperA}.

\textbf{GPU computation.}
Orbit computation is parallelized via Numba CUDA JIT-compiled kernels
(\texttt{scripts/gpu\_attractor\_verification.py}) across $2^8$
threads/block on RTX~4000 Ada (20~GB VRAM), achieving throughput of
${\sim}5 \times 10^6$ iterations/second. Each pipeline--digit-range
combination is tested exhaustively (no sampling).

\textbf{Determinism.}
All computations are deterministic (no random seeds). Basin fractions are
exact rational numbers computed from exhaustive enumeration.

\textbf{Verification hashes.}
For each pipeline and digit range, the SHA-256 hash of the sorted
(endpoint, count) array serves as a verification certificate. Hashes are
reported in Appendix~\ref{app:data}.

\textbf{Conjecture selection.}
The conjectures in Section~\ref{sec:conjectures} were selected from a larger
pool of computationally generated hypotheses using a heuristic prioritization
scheme that weights empirical support, falsification resistance, proof
tractability, novelty, and falsifiability. The weights are manually chosen
(not calibrated or validated); the scheme serves only to guide investigation
priorities and does not constitute a statistical scoring system. All
conjectures stand on their independently stated evidence.

\textbf{Reproducibility.}
All source code, GPU kernels, and raw output data are available at
\url{https://github.com/SYNTRIAD/digit-dynamics}.

%% ===================================================================
\section{Conclusion}\label{sec:conclusion}
%% ===================================================================

We have introduced $\varepsilon$-universality and basin entropy as
quantitative tools for the global dynamics of digit-operation pipelines.
The main finding is a \emph{sharp dichotomy}:

\begin{quote}
\emph{Among the 12~pipelines tested, those mixing contractive and expansive
operations are consistently near-universal ($\varepsilon < 0.04$), while
pipelines combining expansive operations with non-contractive permutations
exhibit rich multi-attractor spectra ($H > 1.5$~bits).}
\end{quote}

The composition lemma (Lemma~\ref{lem:composition}) provides a theoretical
explanation for the first phenomenon, while the conditional Lyapunov theorem
(Theorem~\ref{thm:lyapunov-cond}) gives rigorous convergence guarantees for
the $\mathcal{P} \cup \mathcal{C}$ class.

Open directions include proving Conjectures~\ref{conj:C1}--\ref{conj:C3},
extending the analysis to bases $b \neq 10$, and developing a theory of
\emph{attractor bifurcation} as pipeline parameters vary.

%% ===================================================================
\appendix
\section{Verification Pipeline}\label{app:verification}
%% ===================================================================

\begin{algorithm}[ht]
\caption{Pipeline orbit computation}\label{alg:pipeline}
\begin{algorithmic}[1]
\Require starting value $n_0 \in \mathbb{N}$, pipeline $f = (f_1, \ldots, f_m)$, max iterations $T$
\Ensure endpoint $n$, step count $t$, convergence flag
\State $n \gets n_0$; $\texttt{seen} \gets \{n_0\}$; $t \gets 0$
\While{$t < T$}
  \For{$i = 1$ \textbf{to} $m$}
    \State $n \gets f_i(n)$
  \EndFor
  \State $t \gets t + 1$
  \If{$n \in \texttt{seen}$ \textbf{or} $n = 0$}
    \State \Return $(n, t, \texttt{true})$
  \EndIf
  \State $\texttt{seen} \gets \texttt{seen} \cup \{n\}$
\EndWhile
\State \Return $(n, T, \texttt{false})$
\end{algorithmic}
\end{algorithm}

%% ===================================================================
\section{Dataset and Verification Hashes}\label{app:data}
%% ===================================================================

Full attractor data (pipeline, digit range, attractor set, basin fractions,
convergence profiles) is available at
\url{https://github.com/SYNTRIAD/digit-dynamics/tree/main/data}.

Verification hashes for the four pipelines in Table~\ref{tab:attractors}:

\begin{center}
\footnotesize
\begin{tabular}{@{}ll@{}}
\toprule
Pipeline & SHA-256 (first 16 hex) \\
\midrule
$\text{dp}_4 \to \text{1089}$ & \texttt{c011b908c54b29d8} \\
$\text{1089} \to \text{dp}_4$ & \texttt{cf64b791632661f5} \\
$\kap \to \text{swap}$ & \texttt{ff6d74d4b95bf37c} \\
$\kap \to \sort_\uparrow \to \text{1089} \to \kap$ & \texttt{6c12d71f34c3564b} \\
\bottomrule
\end{tabular}
\end{center}

%% ===================================================================
\begin{thebibliography}{9}

\bibitem{PaperA}
SYNTRIAD Research,
\emph{Fixed points of digit-operation pipelines in arbitrary bases:
algebraic structure and five infinite families},
preprint, February 2026.

\bibitem{Kaprekar1955}
D.~R.~Kaprekar,
\emph{An interesting property of the number 6174},
Scripta Mathematica \textbf{15} (1955), 244--245.

\bibitem{Berger1992}
R.~Berger,
\emph{The Kaprekar routine in general bases},
Fibonacci Quarterly \textbf{30} (1992), no.~4, 349--356.

\bibitem{HardyWright2008}
G.~H.~Hardy and E.~M.~Wright,
\emph{An Introduction to the Theory of Numbers},
6th ed., Oxford University Press, 2008.

\bibitem{Wolfram2002}
S.~Wolfram,
\emph{A New Kind of Science},
Wolfram Media, 2002.

\bibitem{OEIS-A005188}
OEIS Foundation,
\emph{A005188: Narcissistic numbers},
\url{https://oeis.org/A005188}.

\bibitem{OEIS-A006886}
OEIS Foundation,
\emph{A006886: Kaprekar numbers},
\url{https://oeis.org/A006886}.

\bibitem{OEIS-A099009}
OEIS Foundation,
\emph{A099009: Kaprekar fixed points for 6-digit numbers},
\url{https://oeis.org/A099009}.

\bibitem{OEIS-A001232}
OEIS Foundation,
\emph{A001232: Numbers $k$ such that $9k$ = ($k$ written backwards)},
\url{https://oeis.org/A001232}.

\bibitem{Niven1969}
I.~Niven,
\emph{Irrational Numbers},
Mathematical Association of America, 1969.

\bibitem{GuyBook}
R.~K.~Guy,
\emph{Unsolved Problems in Number Theory},
3rd~ed., Springer, 2004.

\bibitem{EverestWard2005}
G.~Everest and T.~Ward,
\emph{An Introduction to Number Theory},
Springer, 2005.

\end{thebibliography}

\end{document}
