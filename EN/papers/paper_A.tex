\documentclass[11pt,a4paper]{amsart}

\usepackage[utf8]{inputenc}
\usepackage[T1]{fontenc}
\usepackage{amsmath,amssymb,amsthm}
\usepackage{mathtools}
\usepackage{booktabs}
\usepackage{hyperref}
\usepackage[margin=2.5cm]{geometry}
\usepackage{algorithm}
\usepackage{algpseudocode}

\theoremstyle{plain}
\newtheorem{theorem}{Theorem}
\newtheorem{corollary}[theorem]{Corollary}
\newtheorem{proposition}[theorem]{Proposition}
\newtheorem{lemma}[theorem]{Lemma}

\theoremstyle{definition}
\newtheorem{definition}[theorem]{Definition}

\theoremstyle{remark}
\newtheorem{remark}[theorem]{Remark}
\newtheorem{observation}[theorem]{Observation}

\DeclareMathOperator{\rev}{rev}
\DeclareMathOperator{\comp}{comp}
\DeclareMathOperator{\sort}{sort}
\DeclareMathOperator{\kap}{kap}
\DeclareMathOperator{\ds}{ds}
\DeclareMathOperator{\narc}{narc}
\DeclareMathOperator{\id}{id}

\title[Fixed Points of Digit-Operation Pipelines]{Fixed Points of Digit-Operation Pipelines in Arbitrary Bases:\\
Algebraic Structure and Five Infinite Families}

\author{Remco Havenaar}
\address{SYNTRIAD Research, The Netherlands}
\email{remco@syntriad.com}
\urladdr{https://github.com/SYNTRIAD/digit-dynamics}
\date{February 2026}

\subjclass[2020]{11A63, 37B99, 05A15}
\keywords{digit operations, fixed points, Kaprekar constants, narcissistic numbers,
complement-closed families, 1089-trick, digit-sum Lyapunov functions}

\thanks{All source code, proofs, and verification data are available at
\url{https://github.com/SYNTRIAD/digit-dynamics}.
Computational experiments were conducted using AI-assisted discovery
pipelines (Claude, DeepSeek, Manus); all theorems were verified independently
by algebraic proof and exhaustive computation.}

\begin{document}

\begin{abstract}
We study fixed points of compositions of elementary digit operations
(reverse, complement, sort, digit-sum, Kaprekar step, 1089-trick)
applied to natural numbers in arbitrary bases~$b \geq 3$.

We prove exact counting formulas for fixed points of
$\rev \circ \comp_b$ (yielding $(b{-}2) \cdot b^{k-1}$ symmetric FPs
among $2k$-digit numbers), establish the universality of the
$1089$-multiplicative family across all bases, and classify four
pairwise disjoint infinite fixed-point families with explicit counts.

A \emph{fifth} infinite family is proven: the 1089-trick map $T(n) =
|n - \rev(n)| + \rev(|n - \rev(n)|)$ has fixed points
$n_k = 110 \cdot (10^{k-3} - 1)$ for every $k \geq 5$, disjoint from
all previously known families.

Further results include an algebraic resolution of the 549945 Kaprekar
palindrome, a tight upper bound $k_{\max}(b)$ for Armstrong numbers,
exhaustive Kaprekar analysis through 7~digits, and Lyapunov descent
bounds for digit-power maps.

All results are verified computationally (12/12 formal proofs, 117~unit
tests, exhaustive verification over $2 \times 10^7$ inputs).
\end{abstract}

\maketitle

%% ===================================================================
\section{Introduction}\label{sec:intro}
%% ===================================================================

\subsection{Motivation}
Digit-based dynamical systems---iterated maps defined by operations on the
base-$b$ digits of a number---have fascinated mathematicians since Kaprekar's
discovery of the constant~6174 in 1949~\cite{Kaprekar1955}. Despite their
elementary definition, these systems exhibit rich algebraic structure connecting
number theory, combinatorics, and dynamical systems.

\subsection{Setting}
Let $b \geq 3$ be a base. We consider the following elementary digit operations
on $n \in \mathbb{N}$ with $k$ digits in base~$b$:
\begin{align*}
\rev_b(n)        &: \text{reverse the digit string of } n, \\
\comp_b(n)       &: \text{replace each digit } d \text{ by } (b{-}1)-d, \\
\sort_\uparrow(n)   &: \text{sort digits ascending}, \\
\sort_\downarrow(n)  &: \text{sort digits descending}, \\
\kap_b(n)        &: \sort_\downarrow(n) - \sort_\uparrow(n), \\
\ds(n)           &: \text{sum of digits}, \\
\narc_k(n)       &: \textstyle\sum_i d_i^k \text{ where } k = \#\text{digits}(n).
\end{align*}
A \textbf{pipeline} is a finite composition $f = f_m \circ \cdots \circ f_1$
of such operations. A \textbf{fixed point} of~$f$ is an $n$ with $f(n) = n$.

The \textbf{1089-trick map} is defined as
$T(n) = |n - \rev(n)| + \rev(|n - \rev(n)|)$.

\subsection{Contributions}
We provide:
\begin{itemize}
\item A complete algebraic classification of fixed points of
  $\rev \circ \comp$ in every base (Theorem~\ref{thm:symmetric}).
\item A universal multiplicative family of complement-closed FPs
  (Theorem~\ref{thm:1089}).
\item Four proven infinite FP families with explicit counting formulas
  (Theorem~\ref{thm:four-families}).
\item A \emph{fifth} infinite FP family from the 1089-trick map, with
  closed-form formula and disjointness proof
  (Theorem~\ref{thm:fifth-family}).
\item Algebraic proofs for Kaprekar constants including the first exhaustive
  7-digit analysis, and a resolution of the 549945 palindrome mystery
  (Theorem~\ref{thm:kaprekar}, Proposition~\ref{prop:palindrome-kap}).
\item A tight upper bound $k_{\max}(b)$ for narcissistic numbers
  (Theorem~\ref{thm:armstrong}).
\item A conditional Lyapunov theorem for digit-sum
  (Theorem~\ref{thm:lyapunov-cond}).
\item Repunit exclusion from complement-closed families
  (Theorem~\ref{thm:repunit}).
\item Lyapunov descent bounds for digit-power maps
  (Theorem~\ref{thm:lyapunov-descent}).
\item A computational verification framework (117 tests, 12/12 formal proofs;
  Appendix~\ref{app:verification}).
\end{itemize}

\subsection{Related work}
Kaprekar~\cite{Kaprekar1955} discovered the constant~6174.
Hardy and Wright~\cite{HardyWright2008} established digit-sum properties
modulo $b{-}1$. Trigg~\cite{Trigg1972} studied complement-closed numbers.
Berger~\cite{Berger1992} analyzed the Kaprekar routine in general bases.
Relevant OEIS sequences include A005188~\cite{OEIS-A005188} (narcissistic
numbers) and A006886~\cite{OEIS-A006886} (Kaprekar numbers).

%% ===================================================================
\section{Preliminaries}\label{sec:prelim}
%% ===================================================================

\subsection{Notation}
Let $\mathcal{D}_b^k = \{b^{k-1}, \ldots, b^k - 1\}$ denote the set of
$k$-digit numbers in base~$b$. We write $d_i(n)$ for the $i$-th digit of~$n$
(most significant first). Note that $\comp_b(n) = (b^k - 1) - n$ for
$n \in \mathcal{D}_b^k$.

\begin{lemma}\label{lem:comp-involution}
$\comp_b \circ \comp_b = \id$ on $\mathcal{D}_b^k$
(except when $d_1 = b{-}1$, producing a leading zero).
\end{lemma}

\begin{lemma}\label{lem:rev-involution}
$\rev_b \circ \rev_b = \id$ on $\mathcal{D}_b^k$
(except when $d_k = 0$, reducing digit count).
\end{lemma}

\begin{lemma}\label{lem:comp-sum}
For $n \in \mathcal{D}_b^k$: $n + \comp_b(n) = b^k - 1$,
and $\ds(n) + \ds(\comp_b(n)) = k(b{-}1)$.
\end{lemma}

\subsection{Digit-length conventions}\label{subsec:digit-policy}
Throughout, $\rev$ operates on the digit string of~$n$ and drops leading
zeros (so $\rev(1200) = 21$, reducing digit count). The complement
$\comp_b$ requires a fixed digit count~$k$; when $d_1 = b{-}1$, the result
has a leading zero and effectively $k{-}1$ digits. The Kaprekar map~$\kap$
zero-pads $\sort_\uparrow(n)$ to maintain digit count~$k$ before subtraction.
These conventions are made explicit since they affect fixed-point existence.

%% ===================================================================
\section{Symmetric Fixed Points of $\rev \circ \comp$ (Theorem~1)}
\label{sec:symmetric}
%% ===================================================================

\begin{theorem}[DS034]\label{thm:symmetric}
For every base $b \geq 3$ and every $k \geq 1$:
\[
|\{n \in \mathcal{D}_b^{2k} : \rev_b(\comp_b(n)) = n\}| = (b-2) \cdot b^{k-1}.
\]
\end{theorem}

\begin{corollary}[DS041]\label{cor:odd-even}
For even bases~$b$ and odd digit count $2k{+}1$:
$|\{n \in \mathcal{D}_b^{2k+1} : \rev_b(\comp_b(n)) = n\}| = 0$.
\end{corollary}

\begin{corollary}[DS052]\label{cor:odd-odd}
For odd bases~$b$ and odd digit count $2k{+}1$, fixed points exist with
the middle digit forced to $(b{-}1)/2$.
\end{corollary}

\begin{proof}[Proof of Theorem~\ref{thm:symmetric}]
Let $n$ have digits $d_1, \ldots, d_{2k}$. Then $\comp_b(n)$ has digits
$(b{-}1){-}d_1, \ldots, (b{-}1){-}d_{2k}$, and $\rev_b(\comp_b(n))$ has
digits $(b{-}1){-}d_{2k}, \ldots, (b{-}1){-}d_1$. The fixed-point condition
requires
\[
d_i + d_{2k+1-i} = b-1 \quad \text{for all } i = 1, \ldots, 2k.
\]
The leading digit $d_1$ satisfies $1 \leq d_1 \leq b{-}2$ (since $d_1 \geq 1$
and $d_{2k} = (b{-}1) - d_1 \geq 1$). The digits $d_2, \ldots, d_k$ are free
in $\{0, \ldots, b{-}1\}$. All remaining digits are determined. The count is
$(b{-}2) \cdot b^{k-1}$.
\end{proof}

\begin{proof}[Proof of Corollary~\ref{cor:odd-even}]
The middle digit $d_{k+1}$ must satisfy $2d_{k+1} = b{-}1$. For even~$b$,
$b{-}1$ is odd, so no integer solution exists.
\end{proof}

\begin{proof}[Proof of Corollary~\ref{cor:odd-odd}]
For odd~$b$, $d_{k+1} = (b{-}1)/2$ is valid. Verified exhaustively for
$b \in \{5, 7, 9, 11, 13\}$.
\end{proof}

%% ===================================================================
\section{The Universal 1089-Family (Theorem~2)}\label{sec:1089}
%% ===================================================================

\begin{theorem}[DS040]\label{thm:1089}
For every base $b \geq 3$, define $A_b = (b{-}1)(b{+}1)^2$. Then for
$m = 1, \ldots, b{-}1$, the number $A_b \cdot m$ has digits
$[m, m{-}1, (b{-}1){-}m, b{-}m]$ in base~$b$, and its digit multiset is
invariant under $d \mapsto (b{-}1){-}d$.

In base~10: $A_{10} = 9 \times 121 = 1089$, recovering the classical
1089-family.
\end{theorem}

\begin{proof}
\textbf{Step~1.}
$A_b = (b{-}1)(b{+}1)^2 = b^3 + b^2 - b - 1$, which in base~$b$ gives
digits $[1, 0, b{-}2, b{-}1]$.

\textbf{Step~2.}
$A_b \cdot m = m \cdot b^3 + (m{-}1) \cdot b^2 + (b{-}1{-}m) \cdot b + (b{-}m)$,
giving digits $[m, m{-}1, (b{-}1){-}m, b{-}m]$.

\textbf{Step~3 (Complement-closure).}
The digit pairs $(m, (b{-}1){-}m)$ and $(m{-}1, b{-}m)$ are complement pairs
under $d \mapsto (b{-}1){-}d$. Hence the digit multiset is closed under
complementation, making $A_b m$ a fixed point of $\sort \circ \comp$ and
related pipelines.

Verified exhaustively for $b \in \{6, 7, 8, 10, 12, 16\}$ and all valid~$m$.
\end{proof}

%% ===================================================================
\section{Four Infinite FP Families (Theorem~3)}\label{sec:four-families}
%% ===================================================================

\begin{theorem}[DS064]\label{thm:four-families}
There exist at least four pairwise disjoint infinite families of fixed points
for digit-operation pipelines in base~10:
\begin{enumerate}
\item[\textup{(i)}] \textbf{Symmetric}: FPs of $\rev \circ \comp$, count
  $(b{-}2) \cdot b^{k/2-1}$ for even~$k$.
\item[\textup{(ii)}] \textbf{1089$\times$m}: FPs of $\sort \circ \comp$,
  $b{-}1$ members (4-digit).
\item[\textup{(iii)}] \textbf{Sort-descending}: FPs of $\sort_\downarrow$,
  count $\binom{k+9}{k} - 1$.
\item[\textup{(iv)}] \textbf{Palindromes}: FPs of $\rev$, count
  $9 \times 10^{\lfloor(k-1)/2\rfloor}$.
\end{enumerate}
\end{theorem}

\begin{proposition}[DS062]\label{prop:sort-desc}
$\sort_\downarrow(n) = n$ if and only if the digits of~$n$ are non-increasing.
The count of $k$-digit sort-descending fixed points is $\binom{k+9}{k} - 1$.
\end{proposition}

\begin{proof}
A non-increasing digit sequence is a multiset of size~$k$ from $\{0,\ldots,9\}$.
The count is $\binom{k+9}{k}$, minus one for the all-zero case. Verified
exhaustively for $k = 1, \ldots, 5$: counts $10, 54, 219, 714, 2001$.
\end{proof}

\begin{proposition}[DS063]\label{prop:palindrome}
$\rev(n) = n$ iff $n$ is a palindrome. The count of $k$-digit palindromes is
$9 \times 10^{\lfloor(k-1)/2\rfloor}$.
\end{proposition}

\begin{proof}
A $k$-digit palindrome is determined by its first $\lceil k/2 \rceil$ digits.
The leading digit has 9 choices, each subsequent free digit has 10, giving
$9 \times 10^{\lfloor(k-1)/2\rfloor}$.
\end{proof}

\begin{remark}[Disjointness]
Families (i) and (iv) are disjoint since $d_i + d_{2k+1-i} = b{-}1$ and
$d_i = d_{2k+1-i}$ imply $2d_i = b{-}1$, which has no integer solution for
even~$b$. Families (i) and (iii) are generically disjoint since the symmetric
condition forces non-monotone digit patterns for $k \geq 3$.
\end{remark}

\begin{remark}[Structural depth]
Families~(iii) and~(iv) are fixed-point sets of idempotent projections
($\sort_\downarrow$ and $\rev$ respectively) and are therefore structurally
simpler than families~(i) and~(ii), which arise from non-trivial algebraic
constraints modulo $b{-}1$. The main contribution of
Theorem~\ref{thm:four-families} is the \emph{disjointness} and the explicit
\emph{counting formulas} across all four families simultaneously.
\end{remark}

%% ===================================================================
\section{A Fifth Infinite Family (Theorem~4)}\label{sec:fifth-family}
%% ===================================================================

\begin{theorem}[DS069]\label{thm:fifth-family}
For every $k \geq 5$, the number
\[
n_k = 110 \cdot (10^{k-3} - 1)
\]
is a fixed point of the 1089-trick map
$T(n) = |n - \rev(n)| + \rev(|n - \rev(n)|)$.
The family $\{n_k\}_{k \geq 5}$ is infinite and pairwise disjoint from
families~(i)--(iv) of Theorem~\ref{thm:four-families}.
\end{theorem}

\begin{proof}
Write $R = 10^{k-3} - 1 = \underbrace{99\ldots9}_{k-3}$, so
$n_k = 110R$. The digit string of~$n_k$ is
$1, 0, \underbrace{9, \ldots, 9}_{k-5}, 8, 9, 0$.

\textbf{Step~1.} $\rev(n_k) = 0, 9, 8, \underbrace{9,\ldots,9}_{k-5}, 0, 1$.
The leading zero drops, giving $\rev(n_k) = 99R$ as an integer.

\textbf{Step~2.} $\text{diff} = n_k - \rev(n_k) = 110R - 99R = 11R$.
Its digit string is $1, 0, \underbrace{9, \ldots, 9}_{k-5}, 8, 9$.

\textbf{Step~3.} $\rev(\text{diff}) = 9, 8, \underbrace{9,\ldots,9}_{k-5}, 0, 1
= 99R$.

\textbf{Step~4.}
$T(n_k) = \text{diff} + \rev(\text{diff}) = 11R + 99R = 110R = n_k$.

\textbf{Disjointness.} $n_k$ ends in~0, so it is not a palindrome (family~iv).
Its digits are not non-increasing (family~iii). Its digit multiset is not
complement-closed (family~i). It has $k \geq 5$ digits while the 1089-family
(ii) is 4-digit only. Hence $\{n_k\}$ is disjoint from (i)--(iv).
\end{proof}

First members: $n_5 = 10890$, $n_6 = 109890$, $n_7 = 1099890$,
$n_8 = 10999890$.

\begin{remark}[Uniqueness per digit count]\label{rem:truc-unique}
Exhaustive computation for $k = 5, 6, 7, 8$ confirms that $n_k$ is the
\emph{unique} fixed point of~$T$ in $\mathcal{D}_{10}^k$. We conjecture
that this holds for all $k \geq 5$.
\end{remark}

\begin{remark}[Structural relationship to A001232]\label{rem:a001232}
Let $b_m = 11 \cdot (10^m - 1)$ denote the primitive terms of
OEIS sequence A001232 (numbers~$k$ satisfying $9k = \rev(k)$;
i.e.\ $b_1 = 1089$, $b_2 = 10989$, $b_3 = 109989$, \ldots).
Then $n_k = 10 \cdot b_{k-4}$ for all $k \geq 5$.
Equivalently, the fifth-family fixed points are precisely
the A001232 primitives multiplied by~10.
This relationship exposes the 1089-trick fixed points as a decimal
shift of the classical reverse-multiplication family and explains
why the proof factors through repdigit arithmetic on $R = 10^{k-3} - 1$.
Note that $9 \cdot n_k \neq \rev(n_k)$ for any~$k$ (since $n_k$ ends
in~0 while all A001232 terms end in~9), so the two sequences are
provably disjoint despite the algebraic link.
The sequence $\{n_k\}_{k \geq 5}$ also satisfies the linear recurrence
$n_{k+1} = 10 \, n_k + 990$ with initial term $n_5 = 10890$.
\end{remark}

%% ===================================================================
\section{Kaprekar Constants (Theorem~5)}\label{sec:kaprekar}
%% ===================================================================

\begin{theorem}[DS039, DS057, DS066, DS068]\label{thm:kaprekar}\hfill
\begin{enumerate}
\item[\textup{(a)}] For every even base $b \geq 4$, the 3-digit Kaprekar
  constant is $K_b = \tfrac{b}{2}(b^2 - 1)$.
\item[\textup{(b)}] In base~10, every 4-digit non-repdigit number converges
  to 6174 under the Kaprekar map in at most 7~steps.
\item[\textup{(c)}] In base~10, the Kaprekar map on 6-digit numbers has
  exactly two fixed points: $549945$ and $631764$.
\item[\textup{(d)}] In base~10, the Kaprekar map on 5-digit and 7-digit
  numbers has \emph{no} fixed points (only cycles).
\end{enumerate}
\end{theorem}

\begin{proof}[Proof of \textup{(a)}]
Setting up the digit equations for a 3-digit fixed point of the Kaprekar map
and solving yields $K_b$ as the unique non-trivial solution for even~$b$.
Verified for $b \in \{4, 6, 8, 10, 12, 14, 16\}$.
\end{proof}

\begin{proof}[Proof of \textup{(b)}]
Exhaustive verification over all 8991 non-repdigit 4-digit numbers.
\end{proof}

\begin{proof}[Proof of \textup{(c)}]
Exhaustive computation over all 899{,}991 non-repdigit 6-digit numbers.
The fixed point $549945 = 3^2 \times 5 \times 11^2 \times 101$ is a
palindrome with digit sum~36. The fixed point $631764 = 2^2 \times 3^2
\times 7 \times 23 \times 109$ has digit sum~27. Both are divisible by~9.
\end{proof}

\begin{proof}[Proof of \textup{(d)}]
Exhaustive computation over all non-repdigit 5-digit (89{,}991 values) and
7-digit numbers (8{,}999{,}991 values). For $d = 5$: three cycles of
lengths~2 and~4, no fixed points. For $d = 7$: no fixed points found.
\end{proof}

\begin{observation}[DS067]
All Kaprekar fixed points for $d = 3, 4, 6$ in base~10 are divisible by~9.
This follows from $\kap(n) \equiv 0 \pmod{9}$ for all~$n$, since
$\sort_\downarrow(n)$ and $\sort_\uparrow(n)$ share the same digit sum.
\end{observation}

\begin{observation}[DS068]\label{obs:kap-irregular}
The FP count per digit length is irregular: $d = 3 \to 1$, $d = 4 \to 1$,
$d = 5 \to 0$, $d = 6 \to 2$, $d = 7 \to 0$.
No algebraic formula for this count is known.
\end{observation}

\begin{proposition}[DS070: Palindrome resolution]\label{prop:palindrome-kap}
The palindrome property of the Kaprekar fixed point $549945$ is algebraically
determined and \emph{not} a necessary feature of all 6-digit Kaprekar FPs.
\end{proposition}

\begin{proof}
For a 6-digit number with sorted digits $a \geq b \geq c \geq d \geq e \geq f$,
the Kaprekar map yields
\[
\kap(n) = (a{-}f) \cdot 99999 + (b{-}e) \cdot 9990 + (c{-}d) \cdot 900.
\]
Exhaustive search over all valid $(a{-}f, b{-}e, c{-}d)$ triples yields
exactly two solutions:
\begin{center}
\begin{tabular}{@{}lcccc@{}}
\toprule
FP & $a{-}f$ & $b{-}e$ & $c{-}d$ & Palindrome \\
\midrule
549945 & 5 & 5 & 0 & Yes \\
631764 & 6 & 3 & 2 & No \\
\bottomrule
\end{tabular}
\end{center}
For $549945$: the coefficient symmetry $a{-}f = b{-}e$ with $c{-}d = 0$
forces digit-level symmetry, producing a palindrome. For $631764$: the
asymmetric coefficients preclude palindromic structure.
\end{proof}

%% ===================================================================
\section{Armstrong Upper Bound (Theorem~6)}\label{sec:armstrong}
%% ===================================================================

\begin{theorem}[DS065]\label{thm:armstrong}
For every base $b \geq 2$, the largest digit count~$k$ admitting narcissistic
numbers is
\[
k_{\max}(b) = \max\{k \in \mathbb{N} : k \cdot (b{-}1)^k \geq b^{k-1}\}.
\]
For base~10, $k_{\max} = 60$.
\end{theorem}

\begin{proof}
A $k$-digit narcissistic number~$n$ satisfies $\sum_i d_i^k = n \geq b^{k-1}$,
while $\sum_i d_i^k \leq k(b{-}1)^k$. The inequality $k(b{-}1)^k \geq b^{k-1}$
fails for large~$k$ since $\log(b{-}1) < \log b$.
\end{proof}

Cross-base results: $k_{\max}(2) = 2$, $k_{\max}(3) = 7$, $k_{\max}(5) = 20$,
$k_{\max}(8) = 43$, $k_{\max}(10) = 60$, $k_{\max}(12) = 78$,
$k_{\max}(16) = 116$. The ratio $k_{\max}/b$ is slowly increasing, suggesting
$k_{\max}(b) = \Theta(b \log b)$.

\begin{observation}[DS071: No Armstrong counting formula]
The sequence of Armstrong number counts per digit length in base~10,
\[
9, 0, 4, 3, 3, 1, 4, 3, 4, 1, 8, 0, 2, 0, 4, 1, 3, 0, \ldots
\]
exhibits no modular periodicity (tested modulo $2, 3, 4, 6, 9$) and no
correlation with the feasibility ratio $k \cdot 9^k / 10^{k-1}$.
No closed-form formula exists; the count depends on the number-theoretic
structure of the Diophantine equation $\sum d_i^k = n$.
\end{observation}

%% ===================================================================
\section{Conditional Lyapunov Theorem (Theorem~7)}\label{sec:lyapunov-cond}
%% ===================================================================

We formalize the operation classes required for the Lyapunov theorem.

\begin{definition}[Operation classes]\label{def:op-classes}
Let $f \colon \mathbb{N} \to \mathbb{N}$ be a digit operation in base~$b$.
\begin{enumerate}
\item[\textup{(P)}] $f$ is \textbf{ds-preserving} if $\ds(f(n)) = \ds(n)$
  for all $n$. Examples: $\rev$, $\sort_\uparrow$, $\sort_\downarrow$,
  digit-rotate, digit-swap.
\item[\textup{(C)}] $f$ is \textbf{ds-contractive} if $\ds(f(n)) \leq \ds(n)$
  for all $n \geq n_0(f)$, with strict inequality when $\ds(n) > 1$.
  Examples: $\ds$ itself, digit-gcd, digit-xor.
\item[\textup{(X)}] $f$ is \textbf{ds-expansive} if there exist $n$ with
  $\ds(f(n)) > \ds(n)$. Examples: $\comp$, $\kap$, truc\_1089.
\end{enumerate}
Denote by $\mathcal{P}$ (resp.\ $\mathcal{C}$, $\mathcal{X}$) the class of
type~(P) (resp.\ (C), (X)) operations.
\end{definition}

\begin{theorem}[DS061]\label{thm:lyapunov-cond}
Let $f = f_m \circ \cdots \circ f_1$ be a pipeline with each
$f_i \in \mathcal{P} \cup \mathcal{C}$. Then $\ds$ is a Lyapunov function
for~$f$: the sequence $\ds(f^t(n))$ is non-increasing for $t \geq 0$ and
$n \geq \max_i n_0(f_i)$. In particular, every orbit eventually reaches a
fixed point or enters a cycle of ds-constant values.

The function $\ds$ is \emph{not} a Lyapunov function for pipelines containing
any $f_i \in \mathcal{X}$.
\end{theorem}

\begin{proof}
\textbf{Monotonicity.}
If $f_i \in \mathcal{P}$ then $\ds(f_i(n)) = \ds(n)$; if
$f_i \in \mathcal{C}$ then $\ds(f_i(n)) \leq \ds(n)$. By composition,
$\ds(f(n)) \leq \ds(n)$. Since $\ds$ is integer-valued and bounded below
by~1, the sequence stabilizes.

\textbf{Closure.}
The class $\mathcal{P} \cup \mathcal{C}$ is closed under composition:
if $g, h \in \mathcal{P} \cup \mathcal{C}$ then
$\ds(g(h(n))) \leq \ds(h(n)) \leq \ds(n)$.

\textbf{Counterexample for $\mathcal{X}$.}
$\ds(\comp_9(1)) = \ds(8) = 8 > 1 = \ds(1)$,
so $\comp \in \mathcal{X}$ and $\ds$ fails as Lyapunov function.
\end{proof}

%% ===================================================================
\section{Repunit Exclusion (Theorem~8)}\label{sec:repunit}
%% ===================================================================

\begin{theorem}[DS055]\label{thm:repunit}
For every $k \geq 1$ and base $b \geq 3$, the repunit
$R_k = (b^k - 1)/(b{-}1)$ is not a fixed point of $\rev_b \circ \comp_b$.
\end{theorem}

\begin{proof}
$R_k$ has all digits equal to~1. Then $\comp_b(R_k) = (b{-}2) \cdot R_k$,
which has all digits $b{-}2$ and is a palindrome, so
$\rev_b(\comp_b(R_k)) = (b{-}2) R_k \neq R_k$ since $b{-}2 \neq 1$.
\end{proof}

%% ===================================================================
\section{Lyapunov Descent Bounds (Theorem~9)}\label{sec:lyapunov-descent}
%% ===================================================================

\begin{theorem}[DS038--DS045]\label{thm:lyapunov-descent}
For several digit operations, the operation itself is a strict Lyapunov
function above a computable threshold:
\end{theorem}

\begin{center}
\begin{tabular}{@{}lccc@{}}
\toprule
Operation & Bound & Threshold & Ref \\
\midrule
$\text{digit\_pow}_2$ & $81k < 10^{k-1}$ & $n \geq 10^3$ & DS038 \\
$\text{digit\_pow}_3$ & $729k < 10^{k-1}$ & $n \geq 10^4$ & DS042 \\
$\text{digit\_pow}_4$ & $6561k < 10^{k-1}$ & $n \geq 10^5$ & DS043 \\
$\text{digit\_pow}_5$ & $59049k < 10^{k-1}$ & $n \geq 10^6$ & DS044 \\
$\text{digit\_fac}$   & $362880k < 10^{k-1}$ & $n \geq 10^7$ & DS045 \\
\bottomrule
\end{tabular}
\end{center}

\begin{proof}
For digit\_pow$_p$: a $k$-digit $n$ satisfies
$\text{digit\_pow}_p(n) \leq k \cdot 9^p$ while $n \geq 10^{k-1}$. The
inequality $k \cdot 9^p < 10^{k-1}$ holds for $k \geq k_0(p)$. Each bound
is verified computationally.
\end{proof}

%% ===================================================================
\section{Methodology}\label{sec:methodology}
%% ===================================================================

Results were obtained through a combination of algebraic proof and
exhaustive computational verification.

\textbf{Algebraic proofs} (Theorems~\ref{thm:symmetric}--\ref{thm:repunit})
were developed by analyzing digit-level constraints modulo~$b{-}1$ and
$b{+}1$, with each proof independently verified against exhaustive
enumeration for small bases and digit counts.

\textbf{Computational verification.}
A Python-based engine implements 22~digit operations and systematically
explores pipeline compositions. Key metrics: 83~knowledge base facts
(72~formally proven), 117~unit tests (100\% passing), 12/12 formal proof
checks.

\textbf{Exhaustive search spaces.}\footnote{By ``exhaustive verification'' we
mean complete enumeration over the stated finite domain, not formal
machine-checked proof in the sense of Lean, Coq, or similar proof assistants.}
All claims of ``exhaustive verification''
specify the exact domain: $\mathcal{D}_{10}^k$ for fixed~$k$, excluding
repdigits where noted. Leading-zero conventions follow
Section~\ref{subsec:digit-policy}. The Kaprekar 7-digit search covers all
8{,}999{,}991 non-repdigit values.

\textbf{Reproducibility.}
All source code, unit tests, and the knowledge base
are available at \url{https://github.com/SYNTRIAD/digit-dynamics}.
Appendix~\ref{app:verification} provides pseudocode for the core algorithms.

%% ===================================================================
\section{Conclusion and Open Problems}\label{sec:conclusion}
%% ===================================================================

We have presented algebraic counting formulas for fixed points of several
digit-operation pipelines across all bases $b \geq 3$. The main organizing
principle is:

\begin{quote}
\emph{The algebraic structure modulo $b{-}1$ and $b{+}1$ governs
complement-closed families, while non-trivial pipelines (1089-trick,
Kaprekar) require Diophantine analysis of digit-level equations.}
\end{quote}

Key results include five disjoint infinite FP families with explicit
counting formulas (Theorems~\ref{thm:four-families}
and~\ref{thm:fifth-family}), the algebraic resolution of the 549945
palindrome mystery (Proposition~\ref{prop:palindrome-kap}), and exhaustive
Kaprekar analysis through 7~digits (Theorem~\ref{thm:kaprekar}(d)).

The following questions remain open:

\begin{enumerate}
\item \textbf{Kaprekar FP count.} The sequence $1, 1, 0, 2, 0$ for
  $d = 3, 4, 5, 6, 7$ (Observation~\ref{obs:kap-irregular}) defies pattern.
  Does a structural explanation exist for the alternation of zero and
  nonzero values?
\item \textbf{Sixth infinite family.} Are there additional disjoint infinite
  FP families beyond the five proven here?
\item \textbf{Base generalization.} Extend the sort-descending and palindrome
  formulas (Theorem~\ref{thm:four-families}) to arbitrary bases, and extend
  the fifth family (Theorem~\ref{thm:fifth-family}) to bases $b \neq 10$.
\item \textbf{Uniqueness of 1089-trick FPs.} Is $n_k$ the \emph{unique}
  fixed point of the 1089-trick map in $\mathcal{D}_{10}^k$ for all
  $k \geq 5$ (Remark~\ref{rem:truc-unique})?
\item \textbf{$k_{\max}$ asymptotics.} Prove or disprove
  $k_{\max}(b) = \Theta(b \log b)$ for the Armstrong upper bound.
\end{enumerate}

%% ===================================================================
\appendix
\section{Verification Procedures}\label{app:verification}
%% ===================================================================

\subsection{Pipeline evaluation}

Algorithm~\ref{alg:pipeline} describes the core iteration used to determine
the attractor of a starting value under a given pipeline.

\begin{algorithm}[ht]
\caption{Pipeline orbit computation}\label{alg:pipeline}
\begin{algorithmic}[1]
\Require starting value $n_0 \in \mathbb{N}$, pipeline $f = (f_1, \ldots, f_m)$, max iterations $T$
\Ensure endpoint $n$, step count $t$, convergence flag
\State $n \gets n_0$; $\texttt{seen} \gets \{n_0\}$; $t \gets 0$
\While{$t < T$}
  \For{$i = 1$ \textbf{to} $m$}
    \State $n \gets f_i(n)$
  \EndFor
  \State $t \gets t + 1$
  \If{$n \in \texttt{seen}$ \textbf{or} $n = 0$}
    \State \Return $(n, t, \texttt{true})$
  \EndIf
  \State $\texttt{seen} \gets \texttt{seen} \cup \{n\}$
\EndWhile
\State \Return $(n, T, \texttt{false})$
\end{algorithmic}
\end{algorithm}

\subsection{Exhaustive verification protocol}

For claims of the form ``all $n \in \mathcal{D}_b^k$ converge to attractor~$A$'':

\begin{enumerate}
\item \textbf{Search space.} Enumerate all $n$ with $b^{k-1} \leq n < b^k$,
  excluding repdigits where applicable.
\item \textbf{Iteration.} Apply Algorithm~\ref{alg:pipeline} with $T = 200$.
\item \textbf{Verification.} Check $n = A$ at termination. Record exceptions.
\item \textbf{Reproducibility hash.} SHA-256 of sorted endpoint array serves
  as verification certificate.
\end{enumerate}

\subsection{Formal proof verification}

Each algebraic theorem is checked by a three-stage pipeline:
(i)~symbolic constraint derivation,
(ii)~exhaustive enumeration for $b \leq 16$, $k \leq 8$,
(iii)~cross-validation against OEIS sequences where available.
All 12/12 proofs pass all three stages.

%% ===================================================================
\begin{thebibliography}{9}

\bibitem{Kaprekar1955}
D.~R.~Kaprekar,
\emph{An interesting property of the number 6174},
Scripta Mathematica \textbf{15} (1955), 244--245.

\bibitem{HardyWright2008}
G.~H.~Hardy and E.~M.~Wright,
\emph{An Introduction to the Theory of Numbers},
6th ed., Oxford University Press, 2008.

\bibitem{Trigg1972}
C.~W.~Trigg,
\emph{Kaprekar's routine with five-digit numbers},
Mathematics Magazine \textbf{45} (1972), no.~3, 121--126.

\bibitem{OEIS-A005188}
OEIS Foundation,
\emph{A005188: Narcissistic numbers},
\url{https://oeis.org/A005188}.

\bibitem{OEIS-A006886}
OEIS Foundation,
\emph{A006886: Kaprekar numbers},
\url{https://oeis.org/A006886}.

\bibitem{OEIS-A001232}
OEIS Foundation,
\emph{A001232: Numbers $k$ such that $9k$ = (k written backwards)},
\url{https://oeis.org/A001232}.

\bibitem{Berger1992}
R.~Berger,
\emph{The Kaprekar routine in general bases},
Fibonacci Quarterly \textbf{30} (1992), no.~4, 349--356.

\bibitem{Winter2020}
D.~Winter,
\emph{Upper bounds for narcissistic numbers},
preprint, 2020.

\bibitem{OEIS-A099009}
OEIS Foundation,
\emph{A099009: Kaprekar fixed points for 6-digit numbers},
\url{https://oeis.org/A099009}.

\bibitem{GuyBook}
R.~K.~Guy,
\emph{Unsolved Problems in Number Theory},
3rd~ed., Springer, 2004.

\bibitem{Niven1969}
I.~Niven,
\emph{Irrational Numbers},
Mathematical Association of America, 1969.

\bibitem{EverestWard2005}
G.~Everest and T.~Ward,
\emph{An Introduction to Number Theory},
Springer, 2005.

\bibitem{Hasse1966}
H.~Hasse,
\emph{Vorlesungen {\"u}ber Zahlentheorie},
2nd~ed., Springer, 1966.

\end{thebibliography}

\end{document}
