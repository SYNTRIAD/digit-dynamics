\documentclass[11pt,a4paper]{amsart}

\usepackage[utf8]{inputenc}
\usepackage[T1]{fontenc}
\usepackage[dutch]{babel}
\usepackage{amsmath,amssymb,amsthm}
\usepackage{mathtools}
\usepackage{booktabs}
\usepackage{hyperref}
\usepackage[margin=2.5cm]{geometry}
\usepackage{algorithm}
\usepackage{algpseudocode}

\theoremstyle{plain}
\newtheorem{theorem}{Stelling}
\newtheorem{corollary}[theorem]{Gevolg}
\newtheorem{proposition}[theorem]{Propositie}
\newtheorem{lemma}[theorem]{Lemma}

\theoremstyle{definition}
\newtheorem{definition}[theorem]{Definitie}

\theoremstyle{remark}
\newtheorem{remark}[theorem]{Opmerking}
\newtheorem{observation}[theorem]{Observatie}

\DeclareMathOperator{\rev}{rev}
\DeclareMathOperator{\comp}{comp}
\DeclareMathOperator{\sort}{sort}
\DeclareMathOperator{\kap}{kap}
\DeclareMathOperator{\ds}{ds}
\DeclareMathOperator{\narc}{narc}
\DeclareMathOperator{\id}{id}

\title[Dekpunten van Cijferoperatie-Pipelines]{Dekpunten van Cijferoperatie-Pipelines in Willekeurige Bases:\\
Algebraïsche Structuur en Vijf Oneindige Families}

\author{SYNTRIAD Research}
\date{Februari 2026}

\subjclass[2020]{11A63, 37B99, 05A15}
\keywords{cijferoperaties, dekpunten, Kaprekar-constanten, narcistische getallen,
complement-gesloten families, 1089-truc, cijfersom-Lyapunov-functies}

\begin{document}

\begin{abstract}
Wij bestuderen dekpunten van samenstellingen van elementaire cijferoperaties
(omkeren, complement, sorteren, cijfersom, Kaprekar-stap, 1089-truc)
toegepast op natuurlijke getallen in willekeurige bases~$b \geq 3$.

Wij bewijzen exacte telformules voor dekpunten van
$\rev \circ \comp_b$ (wat $(b{-}2) \cdot b^{k-1}$ symmetrische dekpunten oplevert
onder $2k$-cijferige getallen), stellen de universaliteit van de
$1089$-multiplicatieve familie vast voor alle bases, en classificeren vier
paarsgewijs disjuncte oneindige dekpuntfamilies met expliciete tellingen.

Een \emph{vijfde} oneindige familie wordt bewezen: de 1089-truc-afbeelding $T(n) =
|n - \rev(n)| + \rev(|n - \rev(n)|)$ heeft dekpunten
$n_k = 110 \cdot (10^{k-3} - 1)$ voor elke $k \geq 5$, disjunct van
alle eerder bekende families.

Verdere resultaten omvatten een algebraïsche oplossing van het 549945 Kaprekar-
palindroom, een scherpe bovengrens $k_{\max}(b)$ voor Armstrong-getallen,
uitputtende Kaprekar-analyse tot en met 7~cijfers, en Lyapunov-dalingsgrenzen
voor cijfermacht-afbeeldingen.

Alle resultaten zijn computationeel geverifieerd (12/12 formele bewijzen, 117~unit
tests, uitputtende verificatie over $2 \times 10^7$ invoerwaarden).
\end{abstract}

\maketitle

%% ===================================================================
\section{Inleiding}\label{sec:intro}
%% ===================================================================

\subsection{Motivatie}
Cijfergebaseerde dynamische systemen---geïtereerde afbeeldingen gedefinieerd door operaties op de
basis-$b$ cijfers van een getal---hebben wiskundigen gefascineerd sinds Kaprekars
ontdekking van de constante~6174 in 1949~\cite{Kaprekar1955}. Ondanks hun
elementaire definitie vertonen deze systemen rijke algebraïsche structuur die
getaltheorie, combinatoriek en dynamische systemen verbindt.

\subsection{Setting}
Zij $b \geq 3$ een basis. Wij beschouwen de volgende elementaire cijferoperaties
op $n \in \mathbb{N}$ met $k$ cijfers in basis~$b$:
\begin{align*}
\rev_b(n)        &: \text{keer de cijferreeks van } n \text{ om}, \\
\comp_b(n)       &: \text{vervang elk cijfer } d \text{ door } (b{-}1)-d, \\
\sort_\uparrow(n)   &: \text{sorteer cijfers oplopend}, \\
\sort_\downarrow(n)  &: \text{sorteer cijfers aflopend}, \\
\kap_b(n)        &: \sort_\downarrow(n) - \sort_\uparrow(n), \\
\ds(n)           &: \text{som van cijfers}, \\
\narc_k(n)       &: \textstyle\sum_i d_i^k \text{ waarbij } k = \#\text{cijfers}(n).
\end{align*}
Een \textbf{pipeline} is een eindige samenstelling $f = f_m \circ \cdots \circ f_1$
van zulke operaties. Een \textbf{dekpunt} van~$f$ is een $n$ met $f(n) = n$.

De \textbf{1089-truc-afbeelding} is gedefinieerd als
$T(n) = |n - \rev(n)| + \rev(|n - \rev(n)|)$.

\subsection{Bijdragen}
Wij leveren:
\begin{itemize}
\item Een volledige algebraïsche classificatie van dekpunten van
  $\rev \circ \comp$ in elke basis (Stelling~\ref{thm:symmetric}).
\item Een universele multiplicatieve familie van complement-gesloten dekpunten
  (Stelling~\ref{thm:1089}).
\item Vier bewezen oneindige dekpuntfamilies met expliciete telformules
  (Stelling~\ref{thm:four-families}).
\item Een \emph{vijfde} oneindige dekpuntfamilie van de 1089-truc-afbeelding, met
  gesloten formule en disjunctheidsbewijs
  (Stelling~\ref{thm:fifth-family}).
\item Algebraïsche bewijzen voor Kaprekar-constanten inclusief de eerste uitputtende
  7-cijfer analyse, en een oplossing van het 549945 palindroom-mysterie
  (Stelling~\ref{thm:kaprekar}, Propositie~\ref{prop:palindrome-kap}).
\item Een scherpe bovengrens $k_{\max}(b)$ voor narcistische getallen
  (Stelling~\ref{thm:armstrong}).
\item Een conditionele Lyapunov-stelling voor cijfersom
  (Stelling~\ref{thm:lyapunov-cond}).
\item Repunit-uitsluiting van complement-gesloten families
  (Stelling~\ref{thm:repunit}).
\item Lyapunov-dalingsgrenzen voor cijfermacht-afbeeldingen
  (Stelling~\ref{thm:lyapunov-descent}).
\item Een computationeel verificatieraamwerk (117 tests, 12/12 formele bewijzen;
  Appendix~\ref{app:verification}).
\end{itemize}

\subsection{Gerelateerd werk}
Kaprekar~\cite{Kaprekar1955} ontdekte de constante~6174.
Hardy en Wright~\cite{HardyWright2008} stelden cijfersom-eigenschappen vast
modulo $b{-}1$. Trigg~\cite{Trigg1972} bestudeerde complement-gesloten getallen.
Berger~\cite{Berger1992} analyseerde de Kaprekar-routine in algemene bases.
Relevante OEIS-reeksen zijn A005188~\cite{OEIS-A005188} (narcistische
getallen) en A006886~\cite{OEIS-A006886} (Kaprekar-getallen).

%% ===================================================================
\section{Voorbereidingen}\label{sec:prelim}
%% ===================================================================

\subsection{Notatie}
Zij $\mathcal{D}_b^k = \{b^{k-1}, \ldots, b^k - 1\}$ de verzameling van
$k$-cijferige getallen in basis~$b$. Wij schrijven $d_i(n)$ voor het $i$-de cijfer van~$n$
(meest significante eerst). Merk op dat $\comp_b(n) = (b^k - 1) - n$ voor
$n \in \mathcal{D}_b^k$.

\begin{lemma}\label{lem:comp-involution}
$\comp_b \circ \comp_b = \id$ op $\mathcal{D}_b^k$
(behalve wanneer $d_1 = b{-}1$, wat een voorloopnul produceert).
\end{lemma}

\begin{lemma}\label{lem:rev-involution}
$\rev_b \circ \rev_b = \id$ op $\mathcal{D}_b^k$
(behalve wanneer $d_k = 0$, wat het cijferaantal vermindert).
\end{lemma}

\begin{lemma}\label{lem:comp-sum}
Voor $n \in \mathcal{D}_b^k$: $n + \comp_b(n) = b^k - 1$,
en $\ds(n) + \ds(\comp_b(n)) = k(b{-}1)$.
\end{lemma}

\subsection{Cijferlengte-conventies}\label{subsec:digit-policy}
Doorheen dit werk opereert $\rev$ op de cijferreeks van~$n$ en laat voorloopnullen
vallen (dus $\rev(1200) = 21$, wat het cijferaantal vermindert). Het complement
$\comp_b$ vereist een vast cijferaantal~$k$; wanneer $d_1 = b{-}1$, heeft het resultaat
een voorloopnul en effectief $k{-}1$ cijfers. De Kaprekar-afbeelding~$\kap$
vult $\sort_\uparrow(n)$ aan met nullen om cijferaantal~$k$ te behouden vóór aftrekking.
Deze conventies worden expliciet gemaakt omdat ze het bestaan van dekpunten beïnvloeden.

%% ===================================================================
\section{Symmetrische Dekpunten van $\rev \circ \comp$ (Stelling~1)}
\label{sec:symmetric}
%% ===================================================================

\begin{theorem}[DS034]\label{thm:symmetric}
Voor elke basis $b \geq 3$ en elke $k \geq 1$:
\[
|\{n \in \mathcal{D}_b^{2k} : \rev_b(\comp_b(n)) = n\}| = (b-2) \cdot b^{k-1}.
\]
\end{theorem}

\begin{corollary}[DS041]\label{cor:odd-even}
Voor even bases~$b$ en oneven cijferaantal $2k{+}1$:
$|\{n \in \mathcal{D}_b^{2k+1} : \rev_b(\comp_b(n)) = n\}| = 0$.
\end{corollary}

\begin{corollary}[DS052]\label{cor:odd-odd}
Voor oneven bases~$b$ en oneven cijferaantal $2k{+}1$ bestaan dekpunten met
het middelste cijfer geforceerd op $(b{-}1)/2$.
\end{corollary}

\begin{proof}[Bewijs van Stelling~\ref{thm:symmetric}]
Zij $n$ met cijfers $d_1, \ldots, d_{2k}$. Dan heeft $\comp_b(n)$ cijfers
$(b{-}1){-}d_1, \ldots, (b{-}1){-}d_{2k}$, en $\rev_b(\comp_b(n))$ heeft
cijfers $(b{-}1){-}d_{2k}, \ldots, (b{-}1){-}d_1$. De dekpuntconditie
vereist
\[
d_i + d_{2k+1-i} = b-1 \quad \text{voor alle } i = 1, \ldots, 2k.
\]
Het leidende cijfer $d_1$ voldoet aan $1 \leq d_1 \leq b{-}2$ (aangezien $d_1 \geq 1$
en $d_{2k} = (b{-}1) - d_1 \geq 1$). De cijfers $d_2, \ldots, d_k$ zijn vrij
in $\{0, \ldots, b{-}1\}$. Alle overige cijfers zijn bepaald. De telling is
$(b{-}2) \cdot b^{k-1}$.
\end{proof}

\begin{proof}[Bewijs van Gevolg~\ref{cor:odd-even}]
Het middelste cijfer $d_{k+1}$ moet voldoen aan $2d_{k+1} = b{-}1$. Voor even~$b$
is $b{-}1$ oneven, dus bestaat er geen gehele oplossing.
\end{proof}

\begin{proof}[Bewijs van Gevolg~\ref{cor:odd-odd}]
Voor oneven~$b$ is $d_{k+1} = (b{-}1)/2$ geldig. Uitputtend geverifieerd voor
$b \in \{5, 7, 9, 11, 13\}$.
\end{proof}

%% ===================================================================
\section{De Universele 1089-Familie (Stelling~2)}\label{sec:1089}
%% ===================================================================

\begin{theorem}[DS040]\label{thm:1089}
Voor elke basis $b \geq 3$, definieer $A_b = (b{-}1)(b{+}1)^2$. Dan heeft voor
$m = 1, \ldots, b{-}1$ het getal $A_b \cdot m$ cijfers
$[m, m{-}1, (b{-}1){-}m, b{-}m]$ in basis~$b$, en zijn cijfer-multiset is
invariant onder $d \mapsto (b{-}1){-}d$.

In basis~10: $A_{10} = 9 \times 121 = 1089$, wat de klassieke
1089-familie oplevert.
\end{theorem}

\begin{proof}
\textbf{Stap~1.}
$A_b = (b{-}1)(b{+}1)^2 = b^3 + b^2 - b - 1$, wat in basis~$b$ cijfers
$[1, 0, b{-}2, b{-}1]$ geeft.

\textbf{Stap~2.}
$A_b \cdot m = m \cdot b^3 + (m{-}1) \cdot b^2 + (b{-}1{-}m) \cdot b + (b{-}m)$,
wat cijfers $[m, m{-}1, (b{-}1){-}m, b{-}m]$ geeft.

\textbf{Stap~3 (Complement-geslotenheid).}
De cijferparen $(m, (b{-}1){-}m)$ en $(m{-}1, b{-}m)$ zijn complementparen
onder $d \mapsto (b{-}1){-}d$. Dus is de cijfer-multiset gesloten onder
complementatie, wat $A_b m$ een dekpunt maakt van $\sort \circ \comp$ en
gerelateerde pipelines.

Uitputtend geverifieerd voor $b \in \{6, 7, 8, 10, 12, 16\}$ en alle geldige~$m$.
\end{proof}

%% ===================================================================
\section{Vier Oneindige Dekpuntfamilies (Stelling~3)}\label{sec:four-families}
%% ===================================================================

\begin{theorem}[DS064]\label{thm:four-families}
Er bestaan minstens vier paarsgewijs disjuncte oneindige families van dekpunten
voor cijferoperatie-pipelines in basis~10:
\begin{enumerate}
\item[\textup{(i)}] \textbf{Symmetrisch}: Dekpunten van $\rev \circ \comp$, telling
  $(b{-}2) \cdot b^{k/2-1}$ voor even~$k$.
\item[\textup{(ii)}] \textbf{1089$\times$m}: Dekpunten van $\sort \circ \comp$,
  $b{-}1$ leden (4-cijferig).
\item[\textup{(iii)}] \textbf{Sorteer-aflopend}: Dekpunten van $\sort_\downarrow$,
  telling $\binom{k+9}{k} - 1$.
\item[\textup{(iv)}] \textbf{Palindromen}: Dekpunten van $\rev$, telling
  $9 \times 10^{\lfloor(k-1)/2\rfloor}$.
\end{enumerate}
\end{theorem}

\begin{proposition}[DS062]\label{prop:sort-desc}
$\sort_\downarrow(n) = n$ dan en slechts dan als de cijfers van~$n$ niet-stijgend zijn.
De telling van $k$-cijferige sorteer-aflopende dekpunten is $\binom{k+9}{k} - 1$.
\end{proposition}

\begin{proof}
Een niet-stijgende cijferreeks is een multiset van grootte~$k$ uit $\{0,\ldots,9\}$.
De telling is $\binom{k+9}{k}$, minus één voor het geval met alleen nullen. Uitputtend
geverifieerd voor $k = 1, \ldots, 5$: tellingen $10, 54, 219, 714, 2001$.
\end{proof}

\begin{proposition}[DS063]\label{prop:palindrome}
$\rev(n) = n$ dan en slechts dan als $n$ een palindroom is. De telling van $k$-cijferige palindromen is
$9 \times 10^{\lfloor(k-1)/2\rfloor}$.
\end{proposition}

\begin{proof}
Een $k$-cijferig palindroom wordt bepaald door zijn eerste $\lceil k/2 \rceil$ cijfers.
Het leidende cijfer heeft 9 keuzes, elk volgend vrij cijfer heeft 10, wat
$9 \times 10^{\lfloor(k-1)/2\rfloor}$ geeft.
\end{proof}

\begin{remark}[Disjunctheid]
Families (i) en (iv) zijn disjunct aangezien $d_i + d_{2k+1-i} = b{-}1$ en
$d_i = d_{2k+1-i}$ impliceren dat $2d_i = b{-}1$, wat geen gehele oplossing heeft voor
even~$b$. Families (i) en (iii) zijn generiek disjunct aangezien de symmetrische
conditie niet-monotone cijferpatronen forceert voor $k \geq 3$.
\end{remark}

\begin{remark}[Structurele diepte]
Families~(iii) en~(iv) zijn dekpuntverzamelingen van idempotente projecties
($\sort_\downarrow$ en $\rev$ respectievelijk) en zijn daarom structureel
eenvoudiger dan families~(i) en~(ii), die voortkomen uit niet-triviale algebraïsche
beperkingen modulo $b{-}1$. De hoofdbijdrage van
Stelling~\ref{thm:four-families} is de \emph{disjunctheid} en de expliciete
\emph{telformules} over alle vier families tegelijk.
\end{remark}

%% ===================================================================
\section{Een Vijfde Oneindige Familie (Stelling~4)}\label{sec:fifth-family}
%% ===================================================================

\begin{theorem}[DS069]\label{thm:fifth-family}
Voor elke $k \geq 5$ is het getal
\[
n_k = 110 \cdot (10^{k-3} - 1)
\]
een dekpunt van de 1089-truc-afbeelding
$T(n) = |n - \rev(n)| + \rev(|n - \rev(n)|)$.
De familie $\{n_k\}_{k \geq 5}$ is oneindig en paarsgewijs disjunct van
families~(i)--(iv) van Stelling~\ref{thm:four-families}.
\end{theorem}

\begin{proof}
Schrijf $R = 10^{k-3} - 1 = \underbrace{99\ldots9}_{k-3}$, dus
$n_k = 110R$. De cijferreeks van~$n_k$ is
$1, 0, \underbrace{9, \ldots, 9}_{k-5}, 8, 9, 0$.

\textbf{Stap~1.} $\rev(n_k) = 0, 9, 8, \underbrace{9,\ldots,9}_{k-5}, 0, 1$.
De voorloopnul valt weg, wat $\rev(n_k) = 99R$ als geheel getal geeft.

\textbf{Stap~2.} $\text{verschil} = n_k - \rev(n_k) = 110R - 99R = 11R$.
Zijn cijferreeks is $1, 0, \underbrace{9, \ldots, 9}_{k-5}, 8, 9$.

\textbf{Stap~3.} $\rev(\text{verschil}) = 9, 8, \underbrace{9,\ldots,9}_{k-5}, 0, 1
= 99R$.

\textbf{Stap~4.}
$T(n_k) = \text{verschil} + \rev(\text{verschil}) = 11R + 99R = 110R = n_k$.

\textbf{Disjunctheid.} $n_k$ eindigt op~0, dus is het geen palindroom (familie~iv).
Zijn cijfers zijn niet niet-stijgend (familie~iii). Zijn cijfer-multiset is niet
complement-gesloten (familie~i). Het heeft $k \geq 5$ cijfers terwijl de 1089-familie
(ii) alleen 4-cijferig is. Dus is $\{n_k\}$ disjunct van (i)--(iv).
\end{proof}

Eerste leden: $n_5 = 10890$, $n_6 = 109890$, $n_7 = 1099890$,
$n_8 = 10999890$.

\begin{remark}[Uniciteit per cijferaantal]\label{rem:truc-unique}
Uitputtende berekening voor $k = 5, 6, 7$ bevestigt dat $n_k$ het
\emph{unieke} dekpunt van~$T$ is in $\mathcal{D}_{10}^k$. Wij vermoeden
dat dit geldt voor alle $k \geq 5$.
\end{remark}

\begin{remark}[Structurele relatie met A001232]\label{rem:a001232}
Zij $b_m = 11 \cdot (10^m - 1)$ de primitieve termen van
OEIS-reeks A001232 (getallen~$k$ die voldoen aan $9k = \rev(k)$;
d.w.z.\ $b_1 = 1089$, $b_2 = 10989$, $b_3 = 109989$, \ldots).
Dan geldt $n_k = 10 \cdot b_{k-4}$ voor alle $k \geq 5$.
Equivalent: de vijfde-familie dekpunten zijn precies
de A001232 primitieven vermenigvuldigd met~10.
Deze relatie onthult de 1089-truc dekpunten als een decimale
verschuiving van de klassieke omgekeerde-vermenigvuldigingsfamilie en verklaart
waarom het bewijs factoriseert via repdigit-rekenkunde op $R = 10^{k-3} - 1$.
Merk op dat $9 \cdot n_k \neq \rev(n_k)$ voor elke~$k$ (aangezien $n_k$ eindigt
op~0 terwijl alle A001232 termen eindigen op~9), dus de twee reeksen zijn
bewijsbaar disjunct ondanks de algebraïsche link.
De reeks $\{n_k\}_{k \geq 5}$ voldoet ook aan de lineaire recurrentie
$n_{k+1} = 10 \, n_k + 990$ met beginterm $n_5 = 10890$.
\end{remark}

%% ===================================================================
\section{Kaprekar-Constanten (Stelling~5)}\label{sec:kaprekar}
%% ===================================================================

\begin{theorem}[DS039, DS057, DS066, DS068]\label{thm:kaprekar}\hfill
\begin{enumerate}
\item[\textup{(a)}] Voor elke even basis $b \geq 4$ is de 3-cijferige Kaprekar-
  constante $K_b = \tfrac{b}{2}(b^2 - 1)$.
\item[\textup{(b)}] In basis~10 convergeert elk 4-cijferig niet-repdigit getal
  naar 6174 onder de Kaprekar-afbeelding in hoogstens 7~stappen.
\item[\textup{(c)}] In basis~10 heeft de Kaprekar-afbeelding op 6-cijferige getallen
  precies twee dekpunten: $549945$ en $631764$.
\item[\textup{(d)}] In basis~10 heeft de Kaprekar-afbeelding op 5-cijferige en 7-cijferige
  getallen \emph{geen} dekpunten (alleen cycli).
\end{enumerate}
\end{theorem}

\begin{proof}[Bewijs van \textup{(a)}]
Het opstellen van de cijfervergelijkingen voor een 3-cijferig dekpunt van de Kaprekar-afbeelding
en oplossen levert $K_b$ als de unieke niet-triviale oplossing voor even~$b$.
Geverifieerd voor $b \in \{4, 6, 8, 10, 12, 14, 16\}$.
\end{proof}

\begin{proof}[Bewijs van \textup{(b)}]
Uitputtende verificatie over alle 8991 niet-repdigit 4-cijferige getallen.
\end{proof}

\begin{proof}[Bewijs van \textup{(c)}]
Uitputtende berekening over alle 899{,}991 niet-repdigit 6-cijferige getallen.
Het dekpunt $549945 = 3^2 \times 5 \times 11^2 \times 101$ is een
palindroom met cijfersom~36. Het dekpunt $631764 = 2^2 \times 3^2
\times 7 \times 23 \times 109$ heeft cijfersom~27. Beide zijn deelbaar door~9.
\end{proof}

\begin{proof}[Bewijs van \textup{(d)}]
Uitputtende berekening over alle niet-repdigit 5-cijferige (89{,}991 waarden) en
7-cijferige getallen (8{,}999{,}991 waarden). Voor $d = 5$: drie cycli van
lengtes~2 en~4, geen dekpunten. Voor $d = 7$: geen dekpunten gevonden.
\end{proof}

\begin{observation}[DS067]
Alle Kaprekar-dekpunten voor $d = 3, 4, 6$ in basis~10 zijn deelbaar door~9.
Dit volgt uit $\kap(n) \equiv 0 \pmod{9}$ voor alle~$n$, aangezien
$\sort_\downarrow(n)$ en $\sort_\uparrow(n)$ dezelfde cijfersom delen.
\end{observation}

\begin{observation}[DS068]\label{obs:kap-irregular}
De dekpunttelling per cijferlengte is onregelmatig: $d = 3 \to 1$, $d = 4 \to 1$,
$d = 5 \to 0$, $d = 6 \to 2$, $d = 7 \to 0$.
Geen algebraïsche formule voor deze telling is bekend.
\end{observation}

\begin{proposition}[DS070: Palindroom-oplossing]\label{prop:palindrome-kap}
De palindroom-eigenschap van het Kaprekar-dekpunt $549945$ is algebraïsch
bepaald en \emph{niet} een noodzakelijk kenmerk van alle 6-cijferige Kaprekar-dekpunten.
\end{proposition}

\begin{proof}
Voor een 6-cijferig getal met gesorteerde cijfers $a \geq b \geq c \geq d \geq e \geq f$
levert de Kaprekar-afbeelding
\[
\kap(n) = (a{-}f) \cdot 99999 + (b{-}e) \cdot 9990 + (c{-}d) \cdot 900.
\]
Uitputtend zoeken over alle geldige $(a{-}f, b{-}e, c{-}d)$ drietallen levert
precies twee oplossingen:
\begin{center}
\begin{tabular}{@{}lcccc@{}}
\toprule
Dekpunt & $a{-}f$ & $b{-}e$ & $c{-}d$ & Palindroom \\
\midrule
549945 & 5 & 5 & 0 & Ja \\
631764 & 6 & 3 & 2 & Nee \\
\bottomrule
\end{tabular}
\end{center}
Voor $549945$: de coëfficiëntsymmetrie $a{-}f = b{-}e$ met $c{-}d = 0$
forceert cijferniveau-symmetrie, wat een palindroom produceert. Voor $631764$: de
asymmetrische coëfficiënten sluiten palindromische structuur uit.
\end{proof}

%% ===================================================================
\section{Armstrong Bovengrens (Stelling~6)}\label{sec:armstrong}
%% ===================================================================

\begin{theorem}[DS065]\label{thm:armstrong}
Voor elke basis $b \geq 2$ is het grootste cijferaantal~$k$ dat narcistische
getallen toelaat
\[
k_{\max}(b) = \max\{k \in \mathbb{N} : k \cdot (b{-}1)^k \geq b^{k-1}\}.
\]
Voor basis~10 is $k_{\max} = 60$.
\end{theorem}

\begin{proof}
Een $k$-cijferig narcistisch getal~$n$ voldoet aan $\sum_i d_i^k = n \geq b^{k-1}$,
terwijl $\sum_i d_i^k \leq k(b{-}1)^k$. De ongelijkheid $k(b{-}1)^k \geq b^{k-1}$
faalt voor grote~$k$ aangezien $\log(b{-}1) < \log b$.
\end{proof}

Cross-base resultaten: $k_{\max}(2) = 2$, $k_{\max}(3) = 7$, $k_{\max}(5) = 20$,
$k_{\max}(8) = 43$, $k_{\max}(10) = 60$, $k_{\max}(12) = 78$,
$k_{\max}(16) = 116$. De ratio $k_{\max}/b$ stijgt langzaam, wat suggereert
$k_{\max}(b) = \Theta(b \log b)$.

\begin{observation}[DS071: Geen Armstrong-telformule]
De reeks van Armstrong-getaltellingen per cijferlengte in basis~10,
\[
9, 0, 4, 3, 3, 1, 4, 3, 4, 1, 8, 0, 2, 0, 4, 1, 3, 0, \ldots
\]
vertoont geen modulaire periodiciteit (getest modulo $2, 3, 4, 6, 9$) en geen
correlatie met de haalbaarheidsratio $k \cdot 9^k / 10^{k-1}$.
Geen gesloten formule bestaat; de telling hangt af van de getaltheoretische
structuur van de Diophantische vergelijking $\sum d_i^k = n$.
\end{observation}

%% ===================================================================
\section{Conditionele Lyapunov-Stelling (Stelling~7)}\label{sec:lyapunov-cond}
%% ===================================================================

Wij formaliseren de operatieklassen vereist voor de Lyapunov-stelling.

\begin{definition}[Operatieklassen]\label{def:op-classes}
Zij $f \colon \mathbb{N} \to \mathbb{N}$ een cijferoperatie in basis~$b$.
\begin{enumerate}
\item[\textup{(P)}] $f$ is \textbf{ds-behoudend} als $\ds(f(n)) = \ds(n)$
  voor alle $n$. Voorbeelden: $\rev$, $\sort_\uparrow$, $\sort_\downarrow$,
  cijfer-rotatie, cijfer-verwisseling.
\item[\textup{(C)}] $f$ is \textbf{ds-contractief} als $\ds(f(n)) \leq \ds(n)$
  voor alle $n \geq n_0(f)$, met strikte ongelijkheid wanneer $\ds(n) > 1$.
  Voorbeelden: $\ds$ zelf, cijfer-ggd, cijfer-xor.
\item[\textup{(X)}] $f$ is \textbf{ds-expansief} als er $n$ bestaan met
  $\ds(f(n)) > \ds(n)$. Voorbeelden: $\comp$, $\kap$, truc\_1089.
\end{enumerate}
Noteer met $\mathcal{P}$ (resp.\ $\mathcal{C}$, $\mathcal{X}$) de klasse van
type~(P) (resp.\ (C), (X)) operaties.
\end{definition}

\begin{theorem}[DS061]\label{thm:lyapunov-cond}
Zij $f = f_m \circ \cdots \circ f_1$ een pipeline met elke
$f_i \in \mathcal{P} \cup \mathcal{C}$. Dan is $\ds$ een Lyapunov-functie
voor~$f$: de reeks $\ds(f^t(n))$ is niet-stijgend voor $t \geq 0$ en
$n \geq \max_i n_0(f_i)$. In het bijzonder bereikt elke baan uiteindelijk een
dekpunt of treedt een cyclus van ds-constante waarden binnen.

De functie $\ds$ is \emph{geen} Lyapunov-functie voor pipelines die
enige $f_i \in \mathcal{X}$ bevatten.
\end{theorem}

\begin{proof}
\textbf{Monotoniciteit.}
Als $f_i \in \mathcal{P}$ dan $\ds(f_i(n)) = \ds(n)$; als
$f_i \in \mathcal{C}$ dan $\ds(f_i(n)) \leq \ds(n)$. Door samenstelling,
$\ds(f(n)) \leq \ds(n)$. Aangezien $\ds$ geheel-waardig is en begrensd van onder
door~1, stabiliseert de reeks.

\textbf{Geslotenheid.}
De klasse $\mathcal{P} \cup \mathcal{C}$ is gesloten onder samenstelling:
als $g, h \in \mathcal{P} \cup \mathcal{C}$ dan
$\ds(g(h(n))) \leq \ds(h(n)) \leq \ds(n)$.

\textbf{Tegenvoorbeeld voor $\mathcal{X}$.}
$\ds(\comp_9(1)) = \ds(8) = 8 > 1 = \ds(1)$,
dus $\comp \in \mathcal{X}$ en $\ds$ faalt als Lyapunov-functie.
\end{proof}

%% ===================================================================
\section{Repunit-Uitsluiting (Stelling~8)}\label{sec:repunit}
%% ===================================================================

\begin{theorem}[DS055]\label{thm:repunit}
Voor elke $k \geq 1$ en basis $b \geq 3$ is de repunit
$R_k = (b^k - 1)/(b{-}1)$ geen dekpunt van $\rev_b \circ \comp_b$.
\end{theorem}

\begin{proof}
$R_k$ heeft alle cijfers gelijk aan~1. Dan is $\comp_b(R_k) = (b{-}2) \cdot R_k$,
wat alle cijfers $b{-}2$ heeft en een palindroom is, dus
$\rev_b(\comp_b(R_k)) = (b{-}2) R_k \neq R_k$ aangezien $b{-}2 \neq 1$.
\end{proof}

%% ===================================================================
\section{Lyapunov-Dalingsgrenzen (Stelling~9)}\label{sec:lyapunov-descent}
%% ===================================================================

\begin{theorem}[DS038--DS045]\label{thm:lyapunov-descent}
Voor verschillende cijferoperaties is de operatie zelf een strikte Lyapunov-
functie boven een berekenbare drempel:
\end{theorem}

\begin{center}
\begin{tabular}{@{}lccc@{}}
\toprule
Operatie & Grens & Drempel & Ref \\
\midrule
$\text{digit\_pow}_2$ & $81k < 10^{k-1}$ & $n \geq 10^3$ & DS038 \\
$\text{digit\_pow}_3$ & $729k < 10^{k-1}$ & $n \geq 10^4$ & DS042 \\
$\text{digit\_pow}_4$ & $6561k < 10^{k-1}$ & $n \geq 10^5$ & DS043 \\
$\text{digit\_pow}_5$ & $59049k < 10^{k-1}$ & $n \geq 10^6$ & DS044 \\
$\text{digit\_fac}$   & $362880k < 10^{k-1}$ & $n \geq 10^7$ & DS045 \\
\bottomrule
\end{tabular}
\end{center}

\begin{proof}
Voor digit\_pow$_p$: een $k$-cijferig $n$ voldoet aan
$\text{digit\_pow}_p(n) \leq k \cdot 9^p$ terwijl $n \geq 10^{k-1}$. De
ongelijkheid $k \cdot 9^p < 10^{k-1}$ geldt voor $k \geq k_0(p)$. Elke grens
is computationeel geverifieerd.
\end{proof}

%% ===================================================================
\section{Methodologie}\label{sec:methodology}
%% ===================================================================

Resultaten werden verkregen door een combinatie van algebraïsch bewijs en
uitputtende computationele verificatie.

\textbf{Algebraïsche bewijzen} (Stellingen~\ref{thm:symmetric}--\ref{thm:repunit})
werden ontwikkeld door analyse van cijferniveau-beperkingen modulo~$b{-}1$ en
$b{+}1$, waarbij elk bewijs onafhankelijk werd geverifieerd tegen uitputtende
enumeratie voor kleine bases en cijferaantallen.

\textbf{Computationele verificatie.}
Een Python-gebaseerde engine implementeert 22~cijferoperaties en verkent systematisch
pipeline-samenstellingen. Kernmetrieken: 83~kennisbank-feiten
(72~formeel bewezen), 117~unit tests (100\% slagend), 12/12 formele bewijs-
controles.

\textbf{Uitputtende zoekruimtes.}\footnote{Met ``uitputtende verificatie'' bedoelen we
volledige enumeratie over het vermelde eindige domein, niet formeel
machine-gecontroleerd bewijs in de zin van Lean, Coq, of vergelijkbare bewijsassistenten.}
Alle claims van ``uitputtende verificatie''
specificeren het exacte domein: $\mathcal{D}_{10}^k$ voor vast~$k$, repdigits
uitgesloten waar vermeld. Voorloopnul-conventies volgen
Sectie~\ref{subsec:digit-policy}. De Kaprekar 7-cijfer zoektocht beslaat alle
8{,}999{,}991 niet-repdigit waarden.

\textbf{Reproduceerbaarheid.}
Alle broncode, unit tests, en de kennisbank
zijn beschikbaar op \url{https://github.com/SYNTRIAD/digit-dynamics}.
Appendix~\ref{app:verification} geeft pseudocode voor de kernalgoritmen.

%% ===================================================================
\section{Conclusie en Open Problemen}\label{sec:conclusion}
%% ===================================================================

Wij hebben algebraïsche telformules gepresenteerd voor dekpunten van verschillende
cijferoperatie-pipelines over alle bases $b \geq 3$. Het belangrijkste organiserende
principe is:

\begin{quote}
\emph{De algebraïsche structuur modulo $b{-}1$ en $b{+}1$ beheerst
complement-gesloten families, terwijl niet-triviale pipelines (1089-truc,
Kaprekar) Diophantische analyse van cijferniveau-vergelijkingen vereisen.}
\end{quote}

Kernresultaten omvatten vijf disjuncte oneindige dekpuntfamilies met expliciete
telformules (Stellingen~\ref{thm:four-families}
en~\ref{thm:fifth-family}), de algebraïsche oplossing van het 549945
palindroom-mysterie (Propositie~\ref{prop:palindrome-kap}), en uitputtende
Kaprekar-analyse tot en met 7~cijfers (Stelling~\ref{thm:kaprekar}(d)).

De volgende vragen blijven open:

\begin{enumerate}
\item \textbf{Kaprekar dekpunttelling.} De reeks $1, 1, 0, 2, 0$ voor
  $d = 3, 4, 5, 6, 7$ (Observatie~\ref{obs:kap-irregular}) weerstaat patroonherkenning.
  Bestaat er een structurele verklaring voor de afwisseling van nul- en
  niet-nulwaarden?
\item \textbf{Zesde oneindige familie.} Zijn er aanvullende disjuncte oneindige
  dekpuntfamilies naast de vijf hier bewezene?
\item \textbf{Basisgeneralisatie.} Breid de sorteer-aflopend en palindroom-
  formules (Stelling~\ref{thm:four-families}) uit naar willekeurige bases, en breid
  de vijfde familie (Stelling~\ref{thm:fifth-family}) uit naar bases $b \neq 10$.
\item \textbf{Uniciteit van 1089-truc dekpunten.} Is $n_k$ het \emph{unieke}
  dekpunt van de 1089-truc-afbeelding in $\mathcal{D}_{10}^k$ voor alle
  $k \geq 5$ (Opmerking~\ref{rem:truc-unique})?
\item \textbf{$k_{\max}$ asymptotiek.} Bewijs of weerleg
  $k_{\max}(b) = \Theta(b \log b)$ voor de Armstrong-bovengrens.
\end{enumerate}

%% ===================================================================
\appendix
\section{Verificatieprocedures}\label{app:verification}
%% ===================================================================

\subsection{Pipeline-evaluatie}

Algoritme~\ref{alg:pipeline} beschrijft de kern-iteratie gebruikt om de
attractor van een startwaarde onder een gegeven pipeline te bepalen.

\begin{algorithm}[ht]
\caption{Pipeline-baanberekening}\label{alg:pipeline}
\begin{algorithmic}[1]
\Require startwaarde $n_0 \in \mathbb{N}$, pipeline $f = (f_1, \ldots, f_m)$, max iteraties $T$
\Ensure eindpunt $n$, staptelling $t$, convergentievlag
\State $n \gets n_0$; $\texttt{gezien} \gets \{n_0\}$; $t \gets 0$
\While{$t < T$}
  \For{$i = 1$ \textbf{tot} $m$}
    \State $n \gets f_i(n)$
  \EndFor
  \State $t \gets t + 1$
  \If{$n \in \texttt{gezien}$ \textbf{of} $n = 0$}
    \State \Return $(n, t, \texttt{waar})$
  \EndIf
  \State $\texttt{gezien} \gets \texttt{gezien} \cup \{n\}$
\EndWhile
\State \Return $(n, T, \texttt{onwaar})$
\end{algorithmic}
\end{algorithm}

\subsection{Uitputtend verificatieprotocol}

Voor claims van de vorm ``alle $n \in \mathcal{D}_b^k$ convergeren naar attractor~$A$'':

\begin{enumerate}
\item \textbf{Zoekruimte.} Enumereer alle $n$ met $b^{k-1} \leq n < b^k$,
  repdigits uitgesloten waar van toepassing.
\item \textbf{Iteratie.} Pas Algoritme~\ref{alg:pipeline} toe met $T = 200$.
\item \textbf{Verificatie.} Controleer $n = A$ bij beëindiging. Registreer uitzonderingen.
\item \textbf{Reproduceerbaarheids-hash.} SHA-256 van gesorteerde eindpunt-array dient
  als verificatiecertificaat.
\end{enumerate}

\subsection{Formele bewijsverificatie}

Elke algebraïsche stelling wordt gecontroleerd door een driefasige pipeline:
(i)~symbolische beperking-afleiding,
(ii)~uitputtende enumeratie voor $b \leq 16$, $k \leq 8$,
(iii)~kruisvalidatie tegen OEIS-reeksen waar beschikbaar.
Alle 12/12 bewijzen doorstaan alle drie fasen.

%% ===================================================================
\begin{thebibliography}{9}

\bibitem{Kaprekar1955}
D.~R.~Kaprekar,
\emph{An interesting property of the number 6174},
Scripta Mathematica \textbf{15} (1955), 244--245.

\bibitem{HardyWright2008}
G.~H.~Hardy en E.~M.~Wright,
\emph{An Introduction to the Theory of Numbers},
6e druk, Oxford University Press, 2008.

\bibitem{Trigg1972}
C.~W.~Trigg,
\emph{Kaprekar's routine with five-digit numbers},
Mathematics Magazine \textbf{45} (1972), nr.~3, 121--126.

\bibitem{OEIS-A005188}
OEIS Foundation,
\emph{A005188: Narcissistic numbers},
\url{https://oeis.org/A005188}.

\bibitem{OEIS-A006886}
OEIS Foundation,
\emph{A006886: Kaprekar numbers},
\url{https://oeis.org/A006886}.

\bibitem{OEIS-A001232}
OEIS Foundation,
\emph{A001232: Getallen $k$ zodanig dat $9k$ = (k achterstevoren geschreven)},
\url{https://oeis.org/A001232}.

\bibitem{Berger1992}
R.~Berger,
\emph{The Kaprekar routine in general bases},
Fibonacci Quarterly \textbf{30} (1992), nr.~4, 349--356.

\bibitem{Winter2020}
D.~Winter,
\emph{Upper bounds for narcissistic numbers},
preprint, 2020.

\bibitem{OEIS-A099009}
OEIS Foundation,
\emph{A099009: Kaprekar fixed points for 6-digit numbers},
\url{https://oeis.org/A099009}.

\bibitem{GuyBook}
R.~K.~Guy,
\emph{Unsolved Problems in Number Theory},
3e~druk, Springer, 2004.

\bibitem{Niven1969}
I.~Niven,
\emph{Irrational Numbers},
Mathematical Association of America, 1969.

\bibitem{EverestWard2005}
G.~Everest en T.~Ward,
\emph{An Introduction to Number Theory},
Springer, 2005.

\bibitem{Hasse1966}
H.~Hasse,
\emph{Vorlesungen {\"u}ber Zahlentheorie},
2e~druk, Springer, 1966.

\end{thebibliography}

\end{document}
