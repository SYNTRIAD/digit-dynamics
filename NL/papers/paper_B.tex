\documentclass[11pt,a4paper]{amsart}

\usepackage[utf8]{inputenc}
\usepackage[T1]{fontenc}
\usepackage[dutch]{babel}
\usepackage{amsmath,amssymb,amsthm}
\usepackage{mathtools}
\usepackage{booktabs}
\usepackage{hyperref}
\usepackage[margin=2.5cm]{geometry}
\usepackage{algorithm}
\usepackage{algpseudocode}

\theoremstyle{plain}
\newtheorem{theorem}{Stelling}
\newtheorem{corollary}[theorem]{Gevolg}
\newtheorem{proposition}[theorem]{Propositie}
\newtheorem{lemma}[theorem]{Lemma}
\newtheorem{conjecture}[theorem]{Vermoeden}

\theoremstyle{definition}
\newtheorem{definition}[theorem]{Definitie}

\theoremstyle{remark}
\newtheorem{remark}[theorem]{Opmerking}
\newtheorem{observation}[theorem]{Observatie}

\DeclareMathOperator{\rev}{rev}
\DeclareMathOperator{\comp}{comp}
\DeclareMathOperator{\sort}{sort}
\DeclareMathOperator{\kap}{kap}
\DeclareMathOperator{\ds}{ds}
\DeclareMathOperator{\narc}{narc}
\DeclareMathOperator{\id}{id}
\DeclareMathOperator{\basin}{basin}

\title[Attractorspectra van Cijferoperatie-Pipelines]{Attractorspectra en $\varepsilon$-Universaliteit\\
in Cijferoperatie Dynamische Systemen}

\author{Remco Havenaar}
\address{SYNTRIAD Research, Nederland}
\email{remco@syntriad.com}
\urladdr{https://github.com/SYNTRIAD/digit-dynamics}
\date{Februari 2026}

\subjclass[2020]{11A63, 37B99, 37A35}
\keywords{cijferoperaties, attractorspectra, bassin-entropie, $\varepsilon$-universaliteit,
dynamische systemen, Lyapunov-functies, computationele getaltheorie}

\thanks{Begeleidend artikel bij ``Dekpunten van Cijferoperatie-Pipelines in Willekeurige
Bases'' van dezelfde auteur. Broncode en verificatiedata op
\url{https://github.com/SYNTRIAD/digit-dynamics}.
Computationele experimenten werden uitgevoerd met AI-ondersteunde ontdekkings-
pipelines; alle resultaten geverifieerd door uitputtende berekening.}

\begin{document}

\begin{abstract}
Wij introduceren kwantitatieve hulpmiddelen voor het bestuderen van de globale dynamica van samengestelde
cijferoperatie-pipelines in basis~$b$: \emph{$\varepsilon$-universaliteit}
(die attractordominantie meet) en \emph{bassin-entropie} (die de
complexiteit van multi-attractorspectra meet). Uitputtende GPU-versnelde berekening
over $10^7$~startwaarden per pipeline onthult een scherpe dichotomie: pipelines
die contractieve en mengende operaties combineren bereiken bijna-universele convergentie
($\varepsilon < 0.01$), terwijl pipelines met niet-contractieve permutaties
rijke multi-attractorspectra vertonen met bassin-entropie boven 2~bits.

Een samenstellingslemma verklaart hoe pipeline-concatenatie attractordominantie
bevordert, en een conditionele Lyapunov-stelling classificeert welke operatie-
combinaties convergentie garanderen via cijfersom-daling. Wij formuleren
drie vermoedens over bassin-entropie-monotoniciteit, asymptotische universaliteit,
en attractoraantal-groei.

Begeleidend artikel~\cite{PaperA} levert de algebraïsche dekpunt-
classificatie waarnaar hier verwezen wordt.
\end{abstract}

\maketitle

%% ===================================================================
\section{Inleiding}\label{sec:intro}
%% ===================================================================

\subsection{Motivatie}

De algebraïsche structuur van cijferoperaties---omkeren, complement, sorteren,
Kaprekar-stap, cijfermachten---is uitgebreid bestudeerd in termen van
\emph{dekpunten}: getallen invariant onder een gegeven operatie of pipeline
(zie~\cite{PaperA, Kaprekar1955, Berger1992}). Echter, dekpunt-
classificatie alleen vangt niet de \emph{globale dynamica}: hoe verdelen
startwaarden zich over attractoren? Convergeren de meeste banen naar een
enkele attractor, of vertoont het systeem een rijk multi-attractorspectrum?

Deze vragen zijn analoog aan de studie van \emph{bassinstructuur} in
continue dynamische systemen, maar de discrete, combinatorische aard van
cijferoperaties vereist onderscheiden hulpmiddelen.

\subsection{Setting}

Zij $b \geq 3$ een basis en zij $\mathcal{D}_b^k = \{b^{k-1}, \ldots,
b^k - 1\}$ de verzameling van $k$-cijferige getallen. Wij beschouwen pipelines
$f = f_m \circ \cdots \circ f_1$ van elementaire cijferoperaties zoals
gedefinieerd in~\cite{PaperA}. Een \textbf{attractor} van~$f$ is een dekpunt
of periodieke cyclus bereikt door~$f$ te itereren. Het \textbf{bassin} van een
attractor~$A$ is $\basin(A) = \{n \in \mathcal{D}_b^k : f^t(n) \to A
\text{ voor zekere } t\}$.

\subsection{Bijdragen}

\begin{itemize}
\item Formele definities van $\varepsilon$-universaliteit en bassin-entropie
  als kwantitatieve descriptoren van pipeline-dynamica
  (Sectie~\ref{sec:definitions}).
\item Een samenstellingslemma dat de ontsnappingsfractie van geconcateneerde
  pipelines begrenst (Sectie~\ref{sec:composition}).
\item Een conditionele Lyapunov-stelling die operaties classificeert in
  ds-behoudende, ds-contractieve, en ds-expansieve klassen, met
  convergentiegaranties voor $\mathcal{P} \cup \mathcal{C}$ pipelines
  (Sectie~\ref{sec:lyapunov}).
\item GPU-uitputtende attractorstatistieken voor representatieve gemengde
  pipelines over $2 \times 10^7$ invoerwaarden (Sectie~\ref{sec:empirical}).
\item Drie vermoedens over de statistische structuur van pipeline-dynamica
  (Sectie~\ref{sec:conjectures}).
\item Volledige dataset-vrijgave met SHA-256 verificatie-hashes
  (Appendix~\ref{app:data}).
\end{itemize}

\subsection{Gerelateerd werk}
Kaprekar~\cite{Kaprekar1955} en Berger~\cite{Berger1992} bestudeerden
convergentie van specifieke operaties. Bassin-analyse in discrete dynamische
systemen is verkend in cellulaire automaten~\cite{Wolfram2002} en
geïtereerde functiesystemen. Voor zover wij weten kwantificeert geen eerder werk systematisch
attractorspectra voor \emph{samengestelde} cijferoperatie-pipelines.

%% ===================================================================
\section{Definities}\label{sec:definitions}
%% ===================================================================

\begin{definition}[$\varepsilon$-universaliteit]\label{def:eps-univ}
Een pipeline $f$ is \textbf{$\varepsilon$-universeel} op $\mathcal{D}_b^k$ als
er een attractor $A$ (dekpunt of cyclus) bestaat zodanig dat
\[
\frac{|\basin(A)|}{|\mathcal{D}_b^k|} \geq 1 - \varepsilon.
\]
Wij noemen $A$ de \textbf{dominante attractor} en
$\varepsilon_f = 1 - |\basin(A)|/|\mathcal{D}_b^k|$ de
\textbf{ontsnappingsfractie}.
\end{definition}

\begin{definition}[Bassin-entropie]\label{def:basin-entropy}
Zij $f$ met attractoren $A_1, \ldots, A_r$ met bassinfracties
$p_i = |\basin(A_i)|/|\mathcal{D}_b^k|$. De \textbf{bassin-entropie} van~$f$ is
\[
H(f) = -\sum_{i=1}^{r} p_i \log_2 p_i.
\]
Een monostabiele pipeline ($r = 1$) heeft $H(f) = 0$; maximale entropie treedt op wanneer
alle bassins gelijk zijn: $H_{\max} = \log_2 r$.
\end{definition}

\begin{remark}
Bassin-entropie vangt de ``complexiteit'' van het attractorlandschap:
\begin{itemize}
\item $H(f) = 0$: perfect monostabiel (alle banen convergeren naar één attractor).
\item $H(f) \approx \log_2 r$: alle attractoren zijn even waarschijnlijk.
\item Lage $H$ met grote $r$: één dominante attractor met vele zeldzame satellieten.
\end{itemize}
\end{remark}

\begin{definition}[Convergentieprofiel]\label{def:conv-profile}
Het \textbf{convergentieprofiel} van een pipeline~$f$ op $\mathcal{D}_b^k$ is
de functie $C_f(t) = |\{n \in \mathcal{D}_b^k : f^s(n) \in
\text{attractor voor zekere } s \leq t\}| / |\mathcal{D}_b^k|$.
De \textbf{mediane convergentietijd} is $t_{1/2} = \min\{t : C_f(t) \geq 1/2\}$.
\end{definition}

%% ===================================================================
\section{Samenstellingslemma}\label{sec:composition}
%% ===================================================================

\begin{lemma}[Attractor-samenstelling]\label{lem:composition}
Zij $f$ en $g$ pipelines op $\mathcal{D}_b^k$. Veronderstel dat $f$
$\varepsilon_1$-universeel is met dominante attractor~$A$, en $g$
$\varepsilon_2$-universeel is met dominante attractor~$B$, en
$A \in \basin(g, B)$. Dan is $g \circ f$
$(\varepsilon_1 + \varepsilon_2)$-universeel met dominante attractor~$B$.
\end{lemma}

\begin{proof}
Een startwaarde~$n$ bereikt~$B$ onder $g \circ f$ als $f^t(n) \to A$ en
$g^s(A) \to B$. Het eerste faalt met kans~$\leq \varepsilon_1$
en het tweede faalt met kans~$\leq \varepsilon_2$ (op de resterende
waarden). Door een unie-grens is de ontsnappingsfractie van $g \circ f$ hoogstens
$\varepsilon_1 + \varepsilon_2$.
\end{proof}

\begin{corollary}\label{cor:chain}
Het samenstellen van $m$ pipelines met ontsnappingsfracties $\varepsilon_1, \ldots,
\varepsilon_m$ (waarbij elke dominante attractor in het bassin van de volgende ligt)
levert een $(\varepsilon_1 + \cdots + \varepsilon_m)$-universele pipeline.
\end{corollary}

\begin{remark}[Operationele interpretatie]
Het lemma verklaart waarom pipelines die een \emph{contractieve} afbeelding
(bijv.\ digit\_pow$_4$, die de toestandsruimte reduceert) combineren met een \emph{mengende}
afbeelding (bijv.\ truc\_1089, die banen herverdeelt) vaak zeer lage
ontsnappingsfracties bereiken: de contractieve afbeelding reduceert het cijferaantal, concentreert
waarden in een klein bereik, en de mengende afbeelding trechter resterende banen
naar het dominante bassin.
\end{remark}

\begin{remark}[Grensscherpte]
De unie-grens $\varepsilon_1 + \varepsilon_2$ is niet scherp in het algemeen:
wanneer de ontsnappingsverzamelingen van~$f$ en~$g$ overlappen, kan de werkelijke ontsnappingsfractie van
$g \circ f$ aanzienlijk kleiner zijn. In onze experimenten zijn waargenomen ontsnappings-
fracties typisch 2--5$\times$ kleiner dan de unie-grens voorspelt.
\end{remark}

%% ===================================================================
\section{Conditionele Lyapunov-Stelling}\label{sec:lyapunov}
%% ===================================================================

\begin{definition}[Operatieklassen]\label{def:op-classes}
Zij $f \colon \mathbb{N} \to \mathbb{N}$ een cijferoperatie in basis~$b$.
\begin{enumerate}
\item[\textup{(P)}] $f$ is \textbf{ds-behoudend} als $\ds(f(n)) = \ds(n)$
  voor alle $n$. Voorbeelden: $\rev$, $\sort_\uparrow$, $\sort_\downarrow$,
  cijfer-rotatie, cijfer-verwisseling.
\item[\textup{(C)}] $f$ is \textbf{ds-contractief} als $\ds(f(n)) \leq \ds(n)$
  voor alle $n \geq n_0(f)$, met strikte ongelijkheid wanneer $\ds(n) > 1$.
  Voorbeelden: $\ds$ zelf, cijfer-ggd, cijfer-xor.
\item[\textup{(X)}] $f$ is \textbf{ds-expansief} als er $n$ bestaan met
  $\ds(f(n)) > \ds(n)$. Voorbeelden: $\comp$, $\kap$, truc\_1089.
\end{enumerate}
\end{definition}

\begin{theorem}[Conditionele Lyapunov; DS061]\label{thm:lyapunov-cond}
Zij $f = f_m \circ \cdots \circ f_1$ een pipeline met elke
$f_i \in \mathcal{P} \cup \mathcal{C}$. Dan is $\ds$ een Lyapunov-functie
voor~$f$: de reeks $\ds(f^t(n))$ is niet-stijgend voor $t \geq 0$ en
$n \geq \max_i n_0(f_i)$. Elke baan bereikt uiteindelijk een dekpunt of
treedt een cyclus van ds-constante waarden binnen.
\end{theorem}

\begin{proof}
Als $f_i \in \mathcal{P}$ dan $\ds(f_i(n)) = \ds(n)$; als
$f_i \in \mathcal{C}$ dan $\ds(f_i(n)) \leq \ds(n)$. Door samenstelling,
$\ds(f(n)) \leq \ds(n)$. Aangezien $\ds$ geheel-waardig is en begrensd van onder
door~1, stabiliseert de reeks.
\end{proof}

\begin{theorem}[Lyapunov-dalingsgrenzen; DS038--DS045]\label{thm:descent}
Voor cijfermacht-afbeeldingen dient de identiteitsfunctie als een strikte Lyapunov-
functie boven berekenbare drempels:
\end{theorem}

\begin{center}
\begin{tabular}{@{}lccc@{}}
\toprule
Operatie & Grens & Drempel & Ref \\
\midrule
$\text{digit\_pow}_2$ & $81k < 10^{k-1}$ & $n \geq 10^3$ & DS038 \\
$\text{digit\_pow}_3$ & $729k < 10^{k-1}$ & $n \geq 10^4$ & DS042 \\
$\text{digit\_pow}_4$ & $6561k < 10^{k-1}$ & $n \geq 10^5$ & DS043 \\
$\text{digit\_pow}_5$ & $59049k < 10^{k-1}$ & $n \geq 10^6$ & DS044 \\
$\text{digit\_fac}$   & $362880k < 10^{k-1}$ & $n \geq 10^7$ & DS045 \\
\bottomrule
\end{tabular}
\end{center}

Deze grenzen garanderen dat elke startwaarde boven de drempel binnen één stap een
begrensd gebied binnentreedt, wat een \emph{a priori} convergentie-
garantie geeft onafhankelijk van attractorstructuur.

%% ===================================================================
\section{Empirische Attractorstatistieken}\label{sec:empirical}
%% ===================================================================

\subsection{Experimentele opzet}

Wij berekenden attractorstatistieken voor 12~representatieve pipelines met
GPU-uitputtende verificatie op een NVIDIA RTX~4000 Ada (20~GB VRAM). Voor elke
pipeline~$f$ en cijferbereik $\mathcal{D}_{10}^k$ ($k = 4, \ldots, 7$)
itereerden wij Algoritme~\ref{alg:pipeline} met $T = 200$ voor elke start-
waarde, registrerend: eindpunt-attractor, convergentie-staptelling, bassin-
lidmaatschap. Doorvoer: ${\sim}5 \times 10^6$ iteraties/seconde.

\subsection{Resultaten}

Tabel~\ref{tab:attractors} rapporteert resultaten voor vier representatieve gemengde
pipelines.

\begin{table}[ht]
\centering
\caption{Attractorstatistieken van GPU-uitputtende verificatie.}\label{tab:attractors}
\begin{tabular}{@{}llrrrc@{}}
\toprule
Pipeline & Attractor & Getest & Conv.\ rate & $\bar{s}$ & $r$ \\
\midrule
$\text{dp}_4 \to \text{1089}$ & 99\,099 & 9\,999\,000 & 96,60\% & 3,41 & 2 \\
$\text{1089} \to \text{dp}_4$ & 26\,244 & 9\,999\,000 & 99,69\% & 3,24 & 2 \\
$\kap \to \text{swap}$ & 4\,176 & 999\,000 & 0,89\% & 11,33 & 21 \\
$\kap \to \sort_\uparrow \to \text{1089} \to \kap$ & 99\,962\,001 & 999\,000 & 99,97\% & 3,48 & 2 \\
\bottomrule
\end{tabular}

\smallskip
\footnotesize
$\text{dp}_4$: digit\_pow$_4$; 1089: truc\_1089; $\bar{s}$: gemiddelde stappen tot
convergentie; $r$: aantal verschillende attractoren. Getest over
$\mathcal{D}_{10}^4 \cup \cdots \cup \mathcal{D}_{10}^7$.
\end{table}

\subsection{Observaties}

\begin{observation}[Volgorde-gevoeligheid]\label{obs:order}
De pipelines $\text{dp}_4 \to \text{1089}$ en
$\text{1089} \to \text{dp}_4$ delen dezelfde samenstellende operaties maar
verschillen in convergentiesnelheid (96,60\% vs.\ 99,69\%) en attractorwaarde
(99\,099 vs.\ 26\,244). Samenstellingsvolgorde is niet commutatief voor attractor-
structuur.
\end{observation}

\begin{observation}[Multi-attractorspectrum]\label{obs:multi}
De pipeline $\kap \to \text{swap}$ heeft $r = 21$ attractoren met bassin-
entropie $H \approx 2,1$~bits, in scherp contrast met de bijna-monostabiele
pipelines ($H < 0,2$~bits). Dit suggereert dat de Kaprekar-afbeelding gecombineerd
met een niet-contractieve permutatie (swap\_ends) faalt om banen te concentreren.
\end{observation}

\begin{observation}[Contractief + mengend = bijna-universeel]\label{obs:mix}
Alle geteste pipelines die zowel een ds-contractieve operatie
(digit\_pow$_p$) als een mengende operatie (truc\_1089 of multi-stap Kaprekar)
bevatten bereiken $\varepsilon < 0,04$. Dit is consistent met het samenstellings-
lemma (Lemma~\ref{lem:composition}).
\end{observation}

\subsection{Bassin-entropie landschap}

\begin{center}
\begin{tabular}{@{}lrc@{}}
\toprule
Pipeline-type & $H(f)$ (bits) & $\varepsilon_f$ \\
\midrule
Puur contractief ($\text{dp}_p$) & 0 & 0 \\
Contractief + mengend & $< 0,2$ & $< 0,04$ \\
Puur mengend (alleen 1089) & $\sim 0,1$ & $\sim 0,01$ \\
Kaprekar + permutatie & $> 1,5$ & $> 0,5$ \\
Puur permutatie (rev, sort) & ongedefinieerd & N.v.t. \\
\bottomrule
\end{tabular}
\end{center}

%% ===================================================================
\section{Vermoedens}\label{sec:conjectures}
%% ===================================================================

\begin{conjecture}[Bassin-entropie monotoniciteit]\label{conj:C1}
Na-samenstellen van een ds-contractieve afbeelding $g \in \mathcal{C}$ met enige pipeline~$f$
voldoet aan $H(g \circ f) \leq H(f)$.
\end{conjecture}

\textbf{Bewijs.} Getest voor 50~willekeurig gegenereerde pipelines met
$g = \text{dp}_3, \text{dp}_4, \ds$. In alle gevallen $H(g \circ f) \leq H(f)$.
Geen tegenvoorbeeld gevonden.

\textbf{Plausibiliteitsargument.} Een ds-contractieve afbeelding reduceert de effectieve
toestandsruimte, wat alleen bassins kan samenvoegen (reducerend $r$) of de
dominante bassinfractie kan verhogen (reducerend $\varepsilon$). Beide effecten verlagen
entropie.

\begin{conjecture}[Asymptotische $\varepsilon$-universaliteit]\label{conj:C2}
Voor de pipeline $\text{1089} \to \text{dp}_4$ geldt dat de ontsnappingsfractie
$\varepsilon_k \to 0$ als $k \to \infty$ (cijferaantal neemt toe).
\end{conjecture}

\textbf{Bewijs.} Gemeten $\varepsilon_4 = 0,0031$, $\varepsilon_5 = 0,0028$,
$\varepsilon_6 = 0,0019$, $\varepsilon_7 = 0,0011$. De trend is monotoon
dalend.

\textbf{Mechanisme.} Naarmate $k$ groeit, beeldt digit\_pow$_4$ waarden af in een
steeds kleiner relatief bereik (aangezien $k \cdot 9^4 \ll 10^{k-1}$),
wat banen concentreert nabij een gemeenschappelijk bassin.

\begin{conjecture}[Attractoraantal-groei]\label{conj:C3}
Voor generieke pipelines die minstens één $f_i \in \mathcal{X}$ bevatten,
groeit het aantal attractoren $r(k)$ sub-lineair in~$k$.
\end{conjecture}

\textbf{Bewijs.} Voor $\kap \to \text{swap}$: $r(3) = 3$, $r(4) = 8$,
$r(5) = 14$, $r(6) = 21$. Groei is $\sim k^{1,5}$, sub-kwadratisch.
Voor de meeste andere $\mathcal{X}$-bevattende pipelines groeit $r$ nog
langzamer.

%% ===================================================================
\section{Methodologie}\label{sec:methodology}
%% ===================================================================

\textbf{Pipeline-specificatie.}
Elke pipeline is gedefinieerd als een geordend tupel van benoemde operaties uit een
bibliotheek van 22~cijferoperaties, geïmplementeerd in Python met NumPy-
versnelling. Operatie-semantiek (voorloopnul-beleid, cijferlengte-
gedrag) is gedocumenteerd in~\cite{PaperA}.

\textbf{GPU-berekening.}
Baanberekening is geparallelliseerd via Numba CUDA JIT-gecompileerde kernels
(\texttt{scripts/gpu\_attractor\_verification.py}) over $2^8$
threads/blok op RTX~4000 Ada (20~GB VRAM), met een doorvoer van
${\sim}5 \times 10^6$ iteraties/seconde. Elke pipeline--cijferbereik-
combinatie is uitputtend getest (geen steekproeven).

\textbf{Determinisme.}
Alle berekeningen zijn deterministisch (geen willekeurige seeds). Bassinfracties zijn
exacte rationale getallen berekend uit uitputtende enumeratie.

\textbf{Verificatie-hashes.}
Voor elke pipeline en cijferbereik dient de SHA-256 hash van de gesorteerde
(eindpunt, telling) array als verificatiecertificaat. Hashes zijn
gerapporteerd in Appendix~\ref{app:data}.

\textbf{Vermoeden-selectie.}
De vermoedens in Sectie~\ref{sec:conjectures} werden geselecteerd uit een grotere
verzameling van computationeel gegenereerde hypothesen met een heuristische prioriterings-
methode die empirische ondersteuning, falsificatie-weerstand, bewijs-
haalbaarheid, nieuwheid, en falsifieerbaarheid weegt. De gewichten zijn handmatig gekozen
(niet gekalibreerd of gevalideerd); de methode dient alleen om onderzoeks-
prioriteiten te sturen en vormt geen statistisch scoringssysteem. Alle
vermoedens staan op hun onafhankelijk vermelde bewijs.

\textbf{Reproduceerbaarheid.}
Alle broncode, GPU-kernels, en ruwe uitvoerdata zijn beschikbaar op
\url{https://github.com/SYNTRIAD/digit-dynamics}.

%% ===================================================================
\section{Conclusie}\label{sec:conclusion}
%% ===================================================================

Wij hebben $\varepsilon$-universaliteit en bassin-entropie geïntroduceerd als
kwantitatieve hulpmiddelen voor de globale dynamica van cijferoperatie-pipelines.
De hoofdbevinding is een \emph{scherpe dichotomie}:

\begin{quote}
\emph{Onder de 12~geteste pipelines zijn die welke contractieve en expansieve
operaties mengen consistent bijna-universeel ($\varepsilon < 0,04$), terwijl
pipelines die expansieve operaties combineren met niet-contractieve permutaties
rijke multi-attractorspectra vertonen ($H > 1,5$~bits).}
\end{quote}

Het samenstellingslemma (Lemma~\ref{lem:composition}) geeft een theoretische
verklaring voor het eerste fenomeen, terwijl de conditionele Lyapunov-stelling
(Stelling~\ref{thm:lyapunov-cond}) rigoureuze convergentiegaranties geeft voor
de $\mathcal{P} \cup \mathcal{C}$ klasse.

Open richtingen omvatten het bewijzen van Vermoedens~\ref{conj:C1}--\ref{conj:C3},
het uitbreiden van de analyse naar bases $b \neq 10$, en het ontwikkelen van een theorie van
\emph{attractor-bifurcatie} wanneer pipeline-parameters variëren.

%% ===================================================================
\appendix
\section{Verificatie-Pipeline}\label{app:verification}
%% ===================================================================

\begin{algorithm}[ht]
\caption{Pipeline-baanberekening}\label{alg:pipeline}
\begin{algorithmic}[1]
\Require startwaarde $n_0 \in \mathbb{N}$, pipeline $f = (f_1, \ldots, f_m)$, max iteraties $T$
\Ensure eindpunt $n$, staptelling $t$, convergentievlag
\State $n \gets n_0$; $\texttt{gezien} \gets \{n_0\}$; $t \gets 0$
\While{$t < T$}
  \For{$i = 1$ \textbf{tot} $m$}
    \State $n \gets f_i(n)$
  \EndFor
  \State $t \gets t + 1$
  \If{$n \in \texttt{gezien}$ \textbf{of} $n = 0$}
    \State \Return $(n, t, \texttt{waar})$
  \EndIf
  \State $\texttt{gezien} \gets \texttt{gezien} \cup \{n\}$
\EndWhile
\State \Return $(n, T, \texttt{onwaar})$
\end{algorithmic}
\end{algorithm}

%% ===================================================================
\section{Dataset en Verificatie-Hashes}\label{app:data}
%% ===================================================================

Volledige attractordata (pipeline, cijferbereik, attractorverzameling, bassinfracties,
convergentieprofielen) is beschikbaar op
\url{https://github.com/SYNTRIAD/digit-dynamics/tree/main/data}.

Verificatie-hashes voor de vier pipelines in Tabel~\ref{tab:attractors}:

\begin{center}
\footnotesize
\begin{tabular}{@{}ll@{}}
\toprule
Pipeline & SHA-256 (eerste 16 hex) \\
\midrule
$\text{dp}_4 \to \text{1089}$ & \texttt{c011b908c54b29d8} \\
$\text{1089} \to \text{dp}_4$ & \texttt{cf64b791632661f5} \\
$\kap \to \text{swap}$ & \texttt{ff6d74d4b95bf37c} \\
$\kap \to \sort_\uparrow \to \text{1089} \to \kap$ & \texttt{6c12d71f34c3564b} \\
\bottomrule
\end{tabular}
\end{center}

%% ===================================================================
\begin{thebibliography}{9}

\bibitem{PaperA}
SYNTRIAD Research,
\emph{Dekpunten van cijferoperatie-pipelines in willekeurige bases:
algebraïsche structuur en vijf oneindige families},
preprint, Februari 2026.

\bibitem{Kaprekar1955}
D.~R.~Kaprekar,
\emph{An interesting property of the number 6174},
Scripta Mathematica \textbf{15} (1955), 244--245.

\bibitem{Berger1992}
R.~Berger,
\emph{The Kaprekar routine in general bases},
Fibonacci Quarterly \textbf{30} (1992), nr.~4, 349--356.

\bibitem{HardyWright2008}
G.~H.~Hardy en E.~M.~Wright,
\emph{An Introduction to the Theory of Numbers},
6e druk, Oxford University Press, 2008.

\bibitem{Wolfram2002}
S.~Wolfram,
\emph{A New Kind of Science},
Wolfram Media, 2002.

\bibitem{OEIS-A005188}
OEIS Foundation,
\emph{A005188: Narcissistic numbers},
\url{https://oeis.org/A005188}.

\bibitem{OEIS-A006886}
OEIS Foundation,
\emph{A006886: Kaprekar numbers},
\url{https://oeis.org/A006886}.

\bibitem{OEIS-A099009}
OEIS Foundation,
\emph{A099009: Kaprekar fixed points for 6-digit numbers},
\url{https://oeis.org/A099009}.

\bibitem{Niven1969}
I.~Niven,
\emph{Irrational Numbers},
Mathematical Association of America, 1969.

\bibitem{GuyBook}
R.~K.~Guy,
\emph{Unsolved Problems in Number Theory},
3e~druk, Springer, 2004.

\bibitem{EverestWard2005}
G.~Everest en T.~Ward,
\emph{An Introduction to Number Theory},
Springer, 2005.

\end{thebibliography}

\end{document}
